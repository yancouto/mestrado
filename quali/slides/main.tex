\errorcontextlines=200
\documentclass[10pt, compress]{beamer}

\usetheme{m}

\usepackage[portuguese]{babel}
\usepackage{../../thesis/code}
\usepackage{booktabs}
\usepackage[scale=2]{ccicons}
\usepackage{minted}
\newcommand{\V}[1]{\mathit{#1}}
\algrenewcommand\algorithmicindent{1em}
\usemintedstyle{trac}
\newcommand{\Oh}{\mathcal{O}}

\usepackage{tikz}
\usepackage{tikz-qtree}
\usetikzlibrary{matrix,backgrounds, decorations.pathreplacing, automata, arrows, positioning}
\tikzset{ edge from parent path = {(\tikzparentnode) -- (\tikzchildnode)}}

\tikzset{onslide/.code args={<#1>#2}{%
  \only<#1>{\pgfkeysalso{#2}} % \pgfkeysalso doesn't change the path
}}
\tikzset{
  invisible/.style={opacity=0},
  visible on/.style={alt={#1{}{invisible}}},
  alt/.code args={<#1>#2#3}{%
    \alt<#1>{\pgfkeysalso{#2}}{\pgfkeysalso{#3}} % \pgfkeysalso doesn't change the path
  },
  blink/.style={onslide={<#1> mLightBrown}},
  appear/.style={visible on=<#1->, blink=#1}
}


\title{Persistência e retroatividade em estruturas de dados}
\subtitle{Yan Soares Couto}
\date{2017}
\author{Orientadora: Cristina Gomes Fernandes}
\institute{Instituto de Matemática e Estatística}


\begin{document}

\maketitle

\begin{frame}[fragile]
  \frametitle{Persistência}

	Uma ED é persistente se, ao realizar uma operação sobre ela, não perdemos sua versão anterior.
	\vfill

	\begin{center}
	\begin{minipage}{0.2\linewidth}
	\begin{tikzpicture}[sibling distance=15pt]
		\Tree [.0
			[.1 [.2 3 [.4 5 ] ] ]
			6
		]
	\end{tikzpicture}
	\end{minipage}
	\begin{minipage}{0.78\linewidth}
	\begin{table}
	\centering
	\begin{subalgorithm}{0.5\linewidth}
		\begin{algorithmic}
			\State $p_0 =$ \func{Stack}{}
			\State $p_1 =$ \func{Push}{p_0, 5}
			\State $p_2 =$ \func{Push}{p_1, 7}
			\State $p_3 =$ \func{Push}{p_2, 6}
			\State $p_4 =$ \func{Pop}{p_2}
			\State \func{Top}{p_3}
			\State $p_5 =$ \func{Push}{p_4, 9}
			\State \func{Top}{p_4}
			\State $p_6 =$ \func{Push}{p_0, 5}
		\end{algorithmic}
	\end{subalgorithm}
	\begin{subalgorithm}{0.27\linewidth}
		\begin{algorithmic}
			\State $p_0:$
			\State $p_1:$ 5
			\State $p_2:$ 5 7
			\State $p_3:$ 5 7 6
			\State $p_4:$ 5
			\State Retorna 6
			\State $p_5:$ 5 9
			\State Retorna 5
			\State $p_6:$ 5
		\end{algorithmic}
	\end{subalgorithm}
	\end{table}
	\end{minipage}
	\end{center}
\end{frame}

\begin{frame}[fragile]
	\frametitle{Pilha persistente}

	Uma pilha implementada usando lista ligada é automaticamente persistente.
	\vfill

	\begin{figure}
		\centering
		\begin{tikzpicture}[sibling distance=25pt, nodes={draw, minimum size=6mm}, edge from parent/.append style={<-, shorten <= 5pt, shorten >= 5pt}, level distance=40pt]
			\Tree [.\node[minimum size=6mm](root){};
				[.\node(p14){5}; [.\node(p2){7}; [.\node(p3){6}; ] ] \node(p5){9}; ]
				\node(p6){5};
			]
			\draw (root.south west) edge (root.north east);
			\draw[->, shorten >= 10pt] node[draw=none, left = 1cm of root]{$p_0$} edge (root);

			\draw[->, shorten >= 10pt] node[draw=none, left = 1cm of p14,yshift=12pt]{$p_1$} edge (p14);
			\draw[->, shorten >= 10pt] node[draw=none, left = 1cm of p14]{$p_4$} edge (p14);
			\draw[->, shorten >= 10pt] node[draw=none, left = 1cm of p2]{$p_2$} edge (p2);
			\draw[->, shorten >= 10pt] node[draw=none, left = 1cm of p3]{$p_3$} edge (p3);
			\draw[->, shorten >= 10pt] node[draw=none, right = 1cm of p5]{$p_5$} edge (p5);
			\draw[->, shorten >= 10pt] node[draw=none, right = 1cm of p6]{$p_6$} edge (p6);

		\end{tikzpicture}
		\caption{Arborescência criada pela sequência de operações do slide anterior.}
	\end{figure}
\end{frame}

\begin{frame}[fragile]
	\frametitle{k-ésimo elemento}
	Determinar o k-ésimo elemento da pilha é equivalente a encontrar o k-ésimo ancestral de um nó da árvore.
	\vfill

	\begin{figure}
		\centering
		\begin{tikzpicture}[nodes={draw, circle, minimum size=5mm}, sibling distance=25pt, edge from parent/.append style={<-, shorten <= 2pt, shorten >= 2pt}, level distance=35pt]
			\Tree [.\node(v1){}; [.\node(v2){}; [.\node(v3){}; [.\node(v4){}; [.\node(v5){}; ] ] [.\node(v6){}; [.\node(v7){}; ] ] ] [.\node(v8){}; ] [.\node(v9){}; ] ] [.\node(v10){}; [.\node(v11){}; ] ] ]

			\tikzstyle{p2}=[->, shorten >= 2pt, shorten <= 2pt, dashed, >=stealth]
			\tikzstyle{p4}=[->, shorten >= 2pt, shorten <= 2pt, dotted, >=stealth]

			\draw[p2] (v3) edge[out=90,in=180] (v1);
			\draw[p2] (v4) edge[out=10,in=200] (v2);
			\draw[p2] (v5) edge[out=130,in=190] (v3);
			\draw[p4] (v5) edge[out=150,in=160] (v1);
			\draw[p2] (v8) edge[out=70,in=250] (v1);
			\draw[p2] (v9) edge[out=80,in=-70] (v1);
			\draw[p2] (v11) edge[out=30,in=20] (v1);
			\draw[p4] (v7) edge[out=0,in=-10] (v1);
			\draw[p2] (v6) edge[out=80,in=240] (v2);
			\draw[p2] (v7) edge[out=50,in=-15] (v3);
		\end{tikzpicture}
		\caption{Exemplo de árvore com jump pointers.}
	\end{figure}
\end{frame}

\begin{frame}[fragile]
	\frametitle{Fila usando pilha}

	Uma fila pode ser representada usando uma pilha e um tamanho.
	\vfill

	$Q = (P, n)$
	\vfill

	$\func{Enqueue}{Q, x} \rightarrow \func{Push}{P, x}$
	\vfill
	$\func{Dequeue}{Q} \rightarrow n = n - 1$
	\vfill
	$\func{K-th}{Q, k} \rightarrow \func{K-th}{P, k}$

\end{frame}

\begin{frame}[fragile]
	\frametitle{Deque persistente}

	É possível implementar uma deque usando a mesma ideia da implementação da pilha.
	\vfill

	\only<1>{
	\begin{table}
		\centering

		\begin{subalgorithm}{0.69\textwidth}
			\begin{algorithmic}

				\State $(\V{first}_0, \V{last}_0) = \Call{Deque}{\null}$
				\State $(\V{first}_1, \V{last}_1) = \Call{PushBack}{(\V{first}_0,\V{last}_0), 3}$
				\State $(\V{first}_2, \V{last}_2) = \Call{PushBack}{(\V{first}_1,\V{last}_1), 4}$
				\State $(\V{first}_3, \V{last}_3) = \Call{PushFront}{(\V{first}_2,\V{last}_2), 2}$
				\State $(\V{first}_4, \V{last}_4) = \Call{PushFront}{(\V{first}_3,\V{last}_3), 1}$
				\State $(\V{first}_5, \V{last}_5) = \Call{PopBack}{(\V{first}_3,\V{last}_3)}$
				\State $(\V{first}_6, \V{last}_6) = \Call{PopBack}{(\V{first}_5,\V{last}_5)}$
				\State $(\V{first}_7, \V{last}_7) = \Call{PushFront}{(\V{first}_6,\V{last}_6), 9}$
				\State $(\V{first}_8, \V{last}_8) = \Call{PopFront}{(\V{first}_6,\V{last}_6)}$
				\State $(\V{first}_9, \V{last}_9) = \Call{PushFront}{(\V{first}_8,\V{last}_8), 6}$

			\end{algorithmic}
		\end{subalgorithm} \vrule
		\begin{subalgorithm}{0.3\textwidth}
			\begin{algorithmic}

				\State $\Rightarrow$
				\State $\Rightarrow$ 3
				\State $\Rightarrow$ 3 4
				\State $\Rightarrow$ 2 3 4
				\State $\Rightarrow$ 1 2 3 4
				\State $\Rightarrow$ 2 3
				\State $\Rightarrow$ 2
				\State $\Rightarrow$ 9 2
				\State $\Rightarrow$
				\State $\Rightarrow$ 6

			\end{algorithmic}
		\end{subalgorithm}
		\caption{Exemplo de uso de uma deque persistente.} \label{ex:deque}
	\end{table}}

	\only<2>{
		Uma deque é representada pelo caminho entre 2 nós.
		\vfill
	\begin{figure}
		\centering
		\begin{tikzpicture}[sibling distance=40pt, nodes={draw, minimum size=6mm}, edge from parent/.append style={<-, shorten <= 5pt, shorten >= 5pt}, level distance=40pt, level 3/.style={sibling distance=80pt}, level 1/.style={sibling distance=80pt}, scale=0.8]

			\Tree [.\node[minimum size=6mm](root){};
			[.\node(v3){3}; [.\node(v2){2}; \node(v1){1}; [.\node(v9){9}; ] ] \node(v4){4}; ]
			\node(v6){6};
			]

			\draw (root.south west) edge (root.north east);
			\draw[->, shorten >= 10pt] node[draw=none, left = 1cm of root,yshift=12pt]{$\V{first}_0$} edge (root);
			\draw[->, shorten >= 10pt] node[draw=none, right = 1cm of root,yshift=12pt]{$\V{last}_0$} edge (root);

			\draw[->, shorten >= 10pt] node[draw=none, left = 1cm of v3,yshift=12pt]{$\V{first}_1$} edge (v3);
			\draw[->, shorten >= 10pt] node[draw=none, right = 1cm of v3,yshift=12pt]{$\V{last}_1$} edge (v3);

			\draw[->, shorten >= 10pt] node[draw=none, left = 1cm of v3]{$\V{first}_2$} edge (v3);
			\draw[->, shorten >= 10pt] node[draw=none, right = 1cm of v4,yshift=12pt]{$\V{last}_2$} edge (v4);

			\draw[->, shorten >= 10pt] node[draw=none, left = 1cm of v2,yshift=12pt]{$\V{first}_3$} edge (v2);
			\draw[->, shorten >= 10pt] node[draw=none, right = 1cm of v4]{$\V{last}_3$} edge (v4);

			\draw[->, shorten >= 10pt] node[draw=none, left = 1cm of v1]{$\V{first}_4$} edge (v1);
			\draw[->, shorten >= 10pt] node[draw=none, right = 1cm of v4,yshift=-12pt]{$\V{last}_4$} edge (v4);

			\draw[->, shorten >= 10pt] node[draw=none, left = 1cm of v2]{$\V{first}_5$} edge (v2);
			\draw[->, shorten >= 10pt] node[draw=none, right = 1cm of v3]{$\V{last}_5$} edge (v3);

			\draw[->, shorten >= 10pt] node[draw=none, left = 1cm of v2,yshift=-12pt]{$\V{first}_6$} edge (v2);
			\draw[->, shorten >= 10pt] node[draw=none, right = 1cm of v2,yshift=12pt]{$\V{last}_6$} edge (v2);

			\draw[->, shorten >= 10pt] node[draw=none, left = 1cm of v9]{$\V{first}_7$} edge (v9);
			\draw[->, shorten >= 10pt] node[draw=none, right = 1cm of v2]{$\V{last}_7$} edge (v2);

			\draw[->, shorten >= 10pt] node[draw=none, left = 1cm of root]{$\V{first}_8$} edge (root);
			\draw[->, shorten >= 10pt] node[draw=none, right = 1cm of root]{$\V{last}_8$} edge (root);

			\draw[->, shorten >= 10pt] node[draw=none, left = 1cm of v6]{$\V{first}_9$} edge (v6);
			\draw[->, shorten >= 10pt] node[draw=none, right = 1cm of v6]{$\V{last}_9$} edge (v6);

		\end{tikzpicture}
		\caption{Arborescência criada pela sequência de operações do exemplo.}
	\end{figure}}
\end{frame}

\begin{frame}[fragile]
	\frametitle{Deque recursiva - Estrutura}

	Uma deque de elementos de~$T$ é composta de:
	\begin{itemize}
		\item Um prefixo de até um elemento de~$T$
		\item Uma deque de elementos de~$T \times T$
		\item Um sufixo de até um elemento de~$T$
	\end{itemize}

	\tikzset{drawnode1/.style={row 2 column #1/.style={nodes={draw}}}}
	\tikzset{drawnode2/.style={row 3 column #1/.style={nodes={draw}}}}
	\tikzset{drawnode3/.style={row 4 column #1/.style={nodes={draw}}}}

	\begin{figure}
		\centering
		\begin{tikzpicture}[
				array/.style={matrix of nodes,nodes={draw=none, minimum size=4mm, anchor=center}, nodes in empty cells, row sep=1cm},
			drawnode1/.list={1, 9, 17},
			drawnode2/.list={2, 3, 9, 15, 16},
			drawnode3/.list={4, 5, 6, 7, 9, 11, 12, 13, 14},
			row 2/.style={row sep=0cm}
		]

			\matrix[array] (array) {
				$\V{prefix}$ &  &  &  & & & & & $\V{center}$ &  & & & & &  &  & $\V{suffix}$ \\[-1cm]
			2 &  &  &  & & & & &  &  & & & & &  &  &  \\
			&  &  &  & & & & &  & & & & &  & 7 & 8 &  \\
			& & & 3 & 4 & 5 & 6 &  &  &  &  &  &  &  & & & \\
		};

			\draw[dashed] (array-2-9.south west) -- (array-3-2.north west);
			\draw[dashed] (array-2-9.south east) -- (array-3-16.north east);
			\draw[dashed] (array-3-9.south west) -- (array-4-4.north west);
			\draw[dashed] (array-3-9.south east) -- (array-4-14.north east);

			%\draw (root.south west) edge (root.north east);
			\newcommand{\dnull}[1] { \path[draw] (#1.south west) -- (#1.north east); }

			\foreach \p in {2-17, 3-2, 3-3, 4-9, 4-11, 4-12, 4-13, 4-14} {\dnull{array-\p}}

			\node[left of=array-2-1, xshift=-.2cm]{Nível 1};
			\node[left of=array-3-1, xshift=-.2cm]{Nível 2};
			\node[left of=array-4-1, xshift=-.2cm]{Nível 3};


		\end{tikzpicture}
		\caption{Deque com elementos (2, 3, 4, 5, 6, 7, 8)}
	\end{figure}
\end{frame}

\begin{frame}[fragile]
	\frametitle{Deque recursiva - Acesso}

	O acesso, e todas as outras operações, são feitos de forma recursiva.
	\vfill

	\begin{algorithmic}[1]
	\Function{Front}{$d$}
		\If{$d.\V{prefix} \neq \Null$}
			\State \Return $d.\V{prefix}$
		\ElsIf{$d.\V{center} = \Null$}
			\State \Return $d.\V{suffix}$
		\Else
			\State \Return \func{Front}{d.\V{center}}$.\V{first}$
		\EndIf
	\EndFunction
	\end{algorithmic}

	\vfill
	As funções do tipo~\textsc{Back} são simétricas às do tipo~\textsc{Front}.
\end{frame}

\begin{frame}[fragile]
	\only<1>{\frametitle{Deque recursiva - Inserção}}
	\only<2>{\frametitle{Deque recursiva - Remoção}}

	Para as operações de modificação, nunca modificamos um nó existente.
	\vfill

	Pelo mesmo motivo que a pilha, a implementação é persistente.
	\vfill

	\only<1>{
	\begin{algorithmic}[1]
	\Function{PushFront}{$d, x$}
		\If{$d = \Null$}
			\State \Return \New \func{Node}{x, \Null, \Null} \label{line:dm:puf1}
		\ElsIf{$d.\V{prefix} = \Null$} \label{line:dm:puf2}
			\State \Return \New \func{Node}{x, d.\V{center}, d.\V{suffix}}\label{line:dm:puf2_1}
		\Else
			{\State \small\Return\!\New\!\func{Node}{\Null, \Call{PushFront}{d.\V{center}, (x, d.\V{prefix})}, d.\V{suffix}}}
		\EndIf
	\EndFunction
	\end{algorithmic}
	}

	\only<2>{
	\begin{algorithmic}[1]
	\Function{PopFront}{$d$}
		\If{$d.\V{prefix} \neq \Null \And d.\V{center} = \Null \And d.\V{suffix} = \Null$}
			\State \Return $(d.\V{prefix}, \Null)$
		\ElsIf{$d.\V{prefix} \neq \Null$}
			\State \Return $(d.\V{prefix}, \New \textsc{Node}({\Null, d.\V{center}, d.\V{suffix}}) )$
		\ElsIf{$d.\V{center} = \Null$}
			\State \Return $(d.\V{suffix}, \Call{Deque}{\null})$
		\Else
			\State $(x, c) = \Call{PopFront}{d.\V{center}}$
			\State \Return $(x.\V{first}, \New \Call{Node}{x.second, c, d.\V{suffix}})$
		\EndIf
	\EndFunction
	\end{algorithmic}
	}

\end{frame}

\begin{frame}[fragile]
	\frametitle{Deque recursiva - K-ésimo}

	Similar a decompor um número em soma de potências de dois.
	\vfill

	\begin{algorithmic}[1]
		\Function{k-th}{$d, k$}
			\If{$k = 1 \And d.\V{prefix} \neq \Null$} 
				\State \Return $d.\V{prefix}$\label{line:kth:if1}
			\EndIf
			\If{$k = d.\V{size} \And d.\V{suffix} \neq \Null$}
				\State \Return $d.\V{suffix}$ \label{line:kth:if2}
			\EndIf
			\If{$d.\V{prefix} \neq \Null$} 
				\State $k = k - 1$ \label{line:kth:if3}
			\EndIf
			\If{$k$ is odd}
				\State \Return $\Call{k-th}{d.\V{center}, \left\lceil\frac{k}{2}\right\rceil}.\V{first}$
			\Else
				\State \Return $\Call{k-th}{d.\V{center}, \left\lceil\frac{k}{2}\right\rceil}.\V{second}$
			\EndIf
		\EndFunction
	\end{algorithmic}

\end{frame}

\begin{frame}[fragile]
	\frametitle{Contadores binários redundantes regulares}

	Um contador binário é \alert{redundante} se cada dígito pode ser 0, 1 ou 2.
	\vfill

	Um contador binário redundante é \alert{regular} se tem 0s entre os 2s e entre o primeiro 2 e o começo.
	\vfill

	É possível adicionar 1 a um contador binário redundante regular mudando~$\Oh(1)$ dígitos.
	\vfill

	\begin{figure}
	\centering

	\begin{tikzpicture}[
	array/.style={matrix of nodes,nodes={draw, minimum size=5mm, anchor=center}, nodes in empty cells},
	column 1/.style={nodes={draw=none}},
	row 1/.style={nodes={draw=none}},
	row 2 column 6/.style={nodes={draw=none}}
	]

	\matrix[array] (array) {
		   &        & $i$ & & &  & & $i$  \\
	Caso 1\ \  & \ldots & 2 & 0/1 & \ldots & \hspace{.8cm}$\Rightarrow$\hspace{.8cm} &
				 \ldots & 0 & 1/2 & \ldots  \\
	 };
	\end{tikzpicture}

	\begin{tikzpicture}[
	array/.style={matrix of nodes,nodes={draw, minimum size=5mm, anchor=center}, nodes in empty cells},
	column 1/.style={nodes={draw=none}},
	row 1/.style={nodes={draw=none}},
	row 2 column 7/.style={nodes={draw=none}}
	]

	\matrix[array] (array) {
		   &        & $i$ & & & $j$ & & & $i$ & & & $j$ \\
	Caso 2\ \  & \ldots & 2 & 0 & \ldots & 2 & \hspace{.8cm}$\Rightarrow$\hspace{.8cm} &
				 \ldots & 0 & 1 & \ldots & 2  \\
	 };
	\end{tikzpicture}

	\begin{tikzpicture}[
	array/.style={matrix of nodes,nodes={draw, minimum size=5mm, anchor=center}, nodes in empty cells},
	column 1/.style={nodes={draw=none}},
	row 1/.style={nodes={draw=none}},
	row 2 column 9/.style={nodes={draw=none}}
	]

	\matrix[array] (array) {
	 & & $i$ & & & & & $j$ & & & $i$ & & & & & $j$ \\
	Caso 3\ \  & \ldots & 2 & 1 & \ldots & 0 & \ldots & 2 & \hspace{.8cm}$\Rightarrow$\hspace{.8cm} &
				 \ldots & 0 & 2 & \ldots & 0 & \ldots & 2 \\
	 };
	\end{tikzpicture}

	\caption{Casos possíveis.}
	\end{figure}

\end{frame}

\begin{frame}[fragile]
	\frametitle{Deque de Kaplan e Tarjan}

	Baseada na deque recursiva, usando a ideia de contadores binários redundantes regulares.
	\vfill

	Os prefixos e sufixos são ``sub-deques'' de até 5 elementos.
	\vfill

	Implementação mais complicada (muita análise de casos).
	\vfill

	\begin{figure}
	\centering

	\begin{tikzpicture}[
	array/.style={matrix of nodes,nodes={draw, minimum size=5mm, anchor=center}, nodes in empty cells},
	row 1/.style={nodes={draw=none}},
	column 1/.style={nodes={draw=none}},
	]

	\matrix[array] (array) {
		Dígito\ \  & 2 & 1 & 0 & 0 & 1 & 2 \\
		Elementos na deque\ \  & 0 & 1 & 2 & 3 & 4 & 5 \\
	 };
	\end{tikzpicture}
	\end{figure}

\end{frame}

\begin{frame}[fragile]
	\frametitle{Técnicas Gerais - Offline}
	Se as operações são todas conhecidas de antemão, é possível resolver o problema com uma busca em profundidade na árvore de versões.
	\vfill

	\centering
	\begin{tikzpicture}[
	nodes = {draw, minimum size = 7mm},
	edge from parent path = {(\tikzparentnode) -- (\tikzchildnode)},
	sibling distance=20pt,
	n/.style = {draw=none},
	gr/.style = {color=gray},
	edge from parent/.append style={-, shorten >= 0, shorten <= 0}
			]
	\Tree
	[.\node(v0){\Call{Stack}{\null}};
		[.\node(v1){\Call{Push}{5}};
			[.\node(v2){\Call{Push}{7}};
				\node(v3){\Call{Push}{6}};
				[.\node(v4){\Call{Pop}{\null}};
					\node(v5){\Call{Push}{9}};
				]
			]
		]
		\node(v6){\Call{Push}{5}};
	]

		\node[gr, above of = v3, xshift=-2.5em, yshift=0.5em](v3t){\Call{Top}{\null}};
		\draw[gr, -, shorten >= 2pt, shorten <= 2pt, dashed] (v3t) -- (v3);
		\node[gr, above of = v4, xshift=2.5em, yshift=0.5em](v4t){\Call{Top}{\null}};
		\draw[gr, -, shorten >= 2pt, shorten <= 2pt, dashed] (v4t) -- (v4);
	\end{tikzpicture}

\end{frame}

\begin{frame}[fragile]
	\frametitle{Técnicas Gerais - Implementação Funcional}
	Para implementar uma estrutura de forma funcional, nunca devemos modificar um nó depois de criá-lo.
	\vfill

	\tikzset{
	nodes={circle, minimum size=6mm, draw},
	n/.style = {draw=none},
	b/.style = {fill=gray!40},
	be/.style = {color=gray},
	edge from parent/.append style={->, shorten >= 2pt, shorten <= 2pt},
	sibling distance=20pt,
	level distance=25pt,
	}
	\begin{figure}
		\centering
		\begin{tikzpicture}
			\Tree [.\node(a){$a$}; [.\node(b){$b$}; \node(d){$d$}; \edge[n];\node[n]{}; ] [.\node(c){$c$}; \node(e){$e$}; \edge[n];\node[n]{}; ] ]
		\end{tikzpicture}
		\begin{tikzpicture}
			\Tree [.\node(a){$a$}; [.\node(b){$b$}; \node(d){$d$}; \edge[n];\node[n]{}; ] [.\node(c){$c$}; \node(e){$e$}; \edge[n];\node[n]{}; ] ]
			\node[above of = a, b](al){$a'$};
			\draw[->, be] (al) -- (b);
			\draw[->, be] (al) -- (c);
		\end{tikzpicture}
		\begin{tikzpicture}
			\Tree [.\node(a){$a$}; [.\node(b){$b$}; \node(d){$d$}; \edge[n];\node[n]{}; ] [.\node(c){$c$}; \node(e){$e$}; \edge[n];\node[n]{}; ] ]
			\node[above of = a, b](al){$a'$};
			\node[above of = b, b](bl){$b'$};
			\draw[->, be] (al) -- (bl);
			\draw[->, be] (al) -- (c);
			\draw[->, be] (bl) to[out=230,in=90] (d);
		\end{tikzpicture}
		\begin{tikzpicture}
			\Tree [.\node(a){$a$}; [.\node(b){$b$}; \node(d){$d$}; \edge[n];\node[b](f){$f$}; ] [.\node(c){$c$}; \node(e){$e$}; \edge[n];\node[n]{}; ] ]
			\node[above of = a, b](al){$a'$};
			\node[above of = b, b](bl){$b'$};
			\draw[->, be] (al) -- (bl);
			\draw[->, be] (al) -- (c);
			\draw[->, be] (bl) to[out=230,in=90] (d);
			\draw[->, be] (bl) to[out=320,in=90] (f);
		\end{tikzpicture}
	\end{figure}
\end{frame}

\begin{frame}[fragile]
	\frametitle{Técnicas gerais - Fat node}

	É possível conseguir persistencia parcial trocando cada campo de cada objeto por um vetor de pares.
	\vfill
	$(i, v)$ --- A $i$-ésima operação de modificação alterou aquele campo para o valor~$v$.
	\vfill

	\begin{tikzpicture}
		\node(l)[anchor=center]{left};
		\node(r)[right = 1.5cm of l,anchor=center]{right};
		\node(v)[above right = .7cm and .75cm of l, anchor=center]{value};
		\draw (1, .375) circle (1.5cm);

		\draw[step=0.5] (3.99, 1.49) grid (7, 2);
		\draw[step=0.5] (3.99, 0.49) grid (7, 1);
		\draw[step=0.5] (3.99, -0.51) grid (7, 0);

		\draw[->, shorten >=3pt] (v) -- (4, 1.75);
		\draw[->, shorten >=3pt] (r) -- (4, 0.75);
		\draw[->, shorten >=3pt] (l) edge[out=-30, in=200] (4, -0.25);

	\end{tikzpicture}
\end{frame}

\begin{frame}[fragile]
	\frametitle{Técnicas gerais - Node copying}

	Assumindo grau de entrada constante, o método Node copying tenta simular a implementação funcional aos poucos.
	\vfill
	Armazenamos um vetor de ponteiros extras em cada nó, e um vetor de ponteiros de volta.
	\vfill
	Ao fazer uma modificação a um vértice, copiamos ele e atualizamos os ponteiros que apontam pra ele.
	\vfill
	Se não houver espaço nos ponteiros extras, copiamos estes nós recursivamente.
\end{frame}

\begin{frame}[fragile]
	\frametitle{Árvore rubro-negra persistente}
	Implementação funcional: Totalmente persistente,~$\Oh(n \lg n)$ tempo e espaço.
	\vfill
	Implementação Node copying: Parcialmente persistente,~$\Oh(n \lg n)$ tempo e~$\Oh(n)$ espaço.
	\vfill

	\begin{tikzpicture}[
	ed/.style = {densely dashed, shorten >= 5pt},
	nodes = {draw, circle, minimum size = 8mm},
	edge from parent path = {(\tikzparentnode) -- (\tikzchildnode)},
	sibling distance=20pt,
	n/.style = {draw=none},
	r/.style = {fill=white},
	b/.style = {fill=gray},
	edge from parent/.append style={-, shorten >= 0, shorten <= 0}
	]
	\Tree [.\node[b]{7}; [.\node[b]{5}; [.\node[r]{1}; \edge[ed];\node[n]{}; \edge[ed];\node[n]{}; ] \edge[ed];[.\node[n]{}; ] ] [.\node[b]{10}; [.\node[r]{8}; \edge[ed];\node[n]{}; \edge[ed];\node[n]{}; ] \edge[ed];[.\node[n]{}; ] ] ]
	\end{tikzpicture}
\end{frame}

\begin{frame}[fragile]
	\frametitle{Retroatividade}
	Uma ED é retroativa se podemos inserir operações em qualquer ponto da história dela.
	\vfill

	\begin{table} \footnotesize
	\begin{subalgorithm}{0.34\textwidth}
	\begin{algorithmic}
		\State $q =$ \func{Queue}{}
		\State $q.\text{Insert}(10, ``\Call{Enqueue}{1}")$
		\State $q.\text{Insert}(3, ``\Call{Enqueue}{2}")$
		\State $q.\text{Insert}(5, ``\Call{Dequeue}{\null}")$
		\State $q.\text{Delete}(5)$
	\end{algorithmic}
	\end{subalgorithm}%
	\vrule
	\begin{subalgorithm}{0.14\textwidth}
	\begin{algorithmic}
		\State $q:$
		\State $q:$ 1
		\State $q:$ 2 1
		\State $q:$ 1
		\State $q:$ 2 1
	\end{algorithmic}
	\end{subalgorithm} \vrule
	\begin{subalgorithm}{0.5\textwidth}
	\begin{algorithmic}
		\State $\emptyset$
		\State $\Call{Enqueue}{1}$
		\State $\Call{Enqueue}{2} \rightarrow \Call{Enqueue}{1}$
		\State $\Call{Enqueue}{2}\!\rightarrow\!\Call{Dequeue}{\null}\!\rightarrow\!\Call{Enqueue}{1}$
		\State $\Call{Enqueue}{2} \rightarrow \Call{Enqueue}{1}$
	\end{algorithmic}
	\end{subalgorithm}%
	\caption{Exemplo de uso de uma fila retroativa. Na esquerda, as operações realizadas, no centro o estado atual da fila, e na direita a sequência de operações, ordenada por tempo.}
	\end{table}%

\end{frame}

\begin{frame}[fragile]
	\frametitle{Fila retroativa}

	A ordem dos elementos na fila é a ordem em que foram inseridos.
	\vfill

	Usamos duas ABBBs modificadas para guardar uma lista de~\textsc{Enqueue}s e uma lista de~\textsc{Dequeue}s.
	\vfill

	\begin{tikzpicture}[
		nodes={circle, draw, minimum size=7.5mm},
		ed/.style={->, shorten >= 3pt, shorten <= 3pt}
	]

		\foreach \i/\t in {1/3,2/5,3/6,4/10,5/13,6/15} {
			\draw (2*\i,0) node(v\i){\t};
		}
		\foreach \i in {1,2,4,5} {
			\pgfmathsetmacro{\j}{int(\i+1)}
			\draw[ed] (v\i) -- (v\j);
		}
		\draw[ed, dashed] (v3) -- (v4);

		\draw (7, 1) node(v){8};

		\draw[ed] (v3) edge[in=-150,out=60] (v);
		\draw[ed] (v) edge[out=-30,in=120] (v4);

	\end{tikzpicture}
\end{frame}

\plain{} {\alert{Planejamento \\[-1ex] \hrulefill \\[-1ex]  Retroatividade} }

\begin{frame}[fragile]
	\frametitle{Resultados Gerais}
	Não é possível tornar qualquer estrutura em retroativa de forma eficiente.
	\vfill

	Se fosse, poderiamos avaliar um polinômio em um ponto de forma rápida.
\end{frame}

\begin{frame}[fragile]
	\frametitle{Pilha e deque}
	Baseada na implementação de pilhas usando vetores.
	\vfill

	Usa ABB que armazena +1's e -1's e responde consultas do tipo:
	\\``Qual último ponto em que a soma do prefixo foi exatamente~$k$?''
	\vfill

	\begin{figure}
	\centering

	\begin{tikzpicture}[
	array/.style={matrix of nodes,nodes={draw, minimum size=5mm, anchor=center}, nodes in empty cells},
	row 1/.style={nodes={draw=none}},
	column 1/.style={nodes={draw=none}},
	]

	\matrix[array] (array) {
		Soma do prefixo\ \  &  1  &  0 &  1 &  2 &  3 &  2 \\
		Operação\ \         &  +1 & -1 & +1 & +1 & +1 & -1 \\
	 };
	\end{tikzpicture}
	\end{figure}
\end{frame}

\begin{frame}[fragile]
	\frametitle{Fila de prioridade}

	A ``vida'' de um elemento em uma fila de prioridade é uma semireta ou um \raisebox{\depth}{\rotatebox{180}{L}}.
	\vfill

	Essas retas não se intersectam.
	\vfill

	\begin{tikzpicture}
		\draw (0, 0) edge[->] node[midway,left]{Valor} (0, 5);
		\draw (0, 0) edge[->] node[midway,below]{Tempo} (10, 0);
		\draw (5, 5) -- (10, 5);
		\draw (1.5, 3.5) -- (10, 3.5);
		\draw (2.5, 3) -- (3, 3) -- (3, 0);
		\draw (8.5, 3) -- (10, 3);
		\draw (8, 2.5) -- (9, 2.5) -- (9, 0);
		\draw (0.5, 2) -- (2, 2) -- (2, 0);
		\draw (4, 2) -- (7.5, 2) -- (7.5, 0);
		\draw (9.5, 2) -- (10, 2);
		\draw (5, 1.5) -- (7, 1.5) -- (7, 0);
		\draw (0.5, 1) -- (1.5, 1) -- (1.5, 0);
		\draw (5, 1) -- (5.5, 1) -- (5.5, 0);
		\draw (6, 0.5) -- (6.5, 0.5) -- (6.5, 0);
		

		\foreach \y in {1,...,5} {
			\draw (0,\y) edge[dotted] (10, \y);
		}
		\foreach \x in {1,...,10} {
			\draw (\x, 0) edge[dotted] (\x, 5);
		}

	\end{tikzpicture}
\end{frame}

\begin{frame}[fragile]
	\frametitle{Union-find}

	Utiliza uma ED chamada Link-Cut Tree.
	\vfill

	Não aceita \textsc{join} entre elementos do mesmo conjunto.
	\vfill

	\begin{center}
	\begin{tikzpicture}[
		every tree node/.style = {draw, circle},
		sibling distance=35pt,
		level distance=40pt,
	]
		\Tree [.\node[orange]{a}; \edge[orange] node[auto]{3}; [.\node[orange]{b}; \edge node[auto]{2}; e \edge[orange] node[auto]{1}; \node[orange]{f}; ]  \edge node[auto]{5}; [.c \edge node[auto]{4}; [.d ] ] ];
	\end{tikzpicture}
	\end{center}
\end{frame}

\plain{} {\alert{Planejamento \\[-1ex] \hrulefill \\[-1ex]  Preliminares}}

\begin{frame}[fragile]
	\frametitle{Ancestrais em árvores}

	Duas técnicas para encontrar Ancestral Comum Mais Profundo e Ancestral de Nível.
	\vfill

	\begin{figure}
		\centering
		\begin{minipage}{0.49\textwidth}
		\begin{tikzpicture}[nodes={draw, circle, minimum size=5mm}, sibling distance=25pt, edge from parent/.append style={<-, shorten <= 2pt, shorten >= 2pt}, level distance=35pt, scale=.6]
			\Tree [.\node(v1){}; [.\node(v2){}; [.\node(v3){}; [.\node(v4){}; [.\node(v5){}; ] ] [.\node(v6){}; [.\node(v7){}; ] ] ] [.\node(v8){}; ] [.\node(v9){}; ] ] [.\node(v10){}; [.\node(v11){}; ] ] ]

			\tikzstyle{p2}=[->, shorten >= 2pt, shorten <= 2pt, dashed, >=stealth]
			\tikzstyle{p4}=[->, shorten >= 2pt, shorten <= 2pt, dotted, >=stealth]

			\draw[p2] (v3) edge[out=90,in=180] (v1);
			\draw[p2] (v4) edge[out=10,in=200] (v2);
			\draw[p2] (v5) edge[out=130,in=190] (v3);
			\draw[p4] (v5) edge[out=150,in=160] (v1);
			\draw[p2] (v8) edge[out=70,in=250] (v1);
			\draw[p2] (v9) edge[out=80,in=-70] (v1);
			\draw[p2] (v11) edge[out=30,in=20] (v1);
			\draw[p4] (v7) edge[out=0,in=-10] (v1);
			\draw[p2] (v6) edge[out=80,in=240] (v2);
			\draw[p2] (v7) edge[out=50,in=-15] (v3);
		\end{tikzpicture}
		\end{minipage}
		\begin{minipage}{0.5\textwidth}
		\begin{tikzpicture}[nodes={draw, circle, minimum size=5mm}, sibling distance=25pt, edge from parent/.append style={<-, shorten <= 2pt, shorten >= 2pt}, level distance=35pt, scale=.6]
			\Tree [.\node(v1){}; [.\node(v2){}; [.\node(v3){}; [.\node(v4){}; [.\node(v5){}; ] ] [.\node(v6){}; [.\node(v7){}; ] ] ] [.\node(v8){}; ] [.\node(v9){}; ] ] [.\node(v10){}; [.\node(v11){}; ] ] ]

			\tikzstyle{p2}=[->, shorten >= 2pt, shorten <= 2pt, dashed, >=stealth]
			\tikzstyle{p4}=[->, shorten >= 2pt, shorten <= 2pt, dotted, >=stealth]

			\draw[p2] (v2) edge[bend left] (v1);
			\draw[p2] (v4) edge[in=170,out=90] (v1);
			\draw[p2] (v10) edge[bend right] (v1);
			\draw[p2] (v11) edge[bend right] (v10);
			\draw[p2] (v9) edge[bend right] (v2);
			\draw[p2] (v3) edge[bend left] (v2);
			\draw[p2] (v8) edge[bend left] (v2);
			\draw[p2] (v5) edge[bend left] (v4);
			\draw[p2] (v7) edge[bend right] (v6);
			\draw[p2] (v6) edge[in=-90,out=10] (v1);

		\end{tikzpicture}
		\end{minipage}
	\end{figure}

\end{frame}

\begin{frame}[fragile]
	\frametitle{ABBs modificadas}

	Explicar ABBs usadas em fila, pilha e deque retroativas.
	\vfill

	\begin{figure}
		\centering
		\begin{tikzpicture}[every tree node/.style={draw, circle}, sibling distance=25pt]
			\Tree [.9 [.5 [.3 1 1 ] 1 ] [.3 1 1 ] ];
		\end{tikzpicture}
		\caption{ABB que armazena o tamanho da subárvore em cada nó.}
	\end{figure}

\end{frame}

\plain{}{\alert{Obrigado}}

\end{document}
