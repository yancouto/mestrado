\documentclass[quali.tex]{subfile}

\begin{document}

\section{Planejamento} \label{sec:planejamento}

\subsection{Retroatividade}

A principal fonte de estudo sobre retroatividade é o artigo de Demaine, Iacono e Langerman~\cite{DemaineIL2007}, onde o tópico é inicialmente apresentado. Alguns dos tópicos discutidos são

\subsubsection*{Resultados gerais}
	Os autores demonstram que retroatividade não é ``fácil'', já que não é possível desenvolver um método para tornar qualquer estrutura em retroativa de forma eficiente.
\subsubsection*{Pilha e deque}
	A implementação de ambas estruturas é parecida, e usa ABBs modificadas para responder consultas sobre somas acumuladas. Em particular, em uma ABB com apenas $+1$'s e $-1$'s, é necessário saber o maior prefixo de elementos que tem soma~$x$. Já implementei esta estrutura.
\subsubsection*{Union-find}
	O artigo descreve uma implementação para a versão retroativa dessa estrutura, na qual é necessário utilizar uma estrutura de dados chamada Link-Cut Tree.
\subsubsection*{Fila de prioridade}
	O artigo descreve uma implementação parcialmente retroativa que usa também ABBs modificadas, mas a teoria por trás desta implementação é mais complicada e teremos que despender um tempo maior para compreendê-la e explicá-la totalmente. \\

Estudei todos estes tópicos pelo artigo, mas algumas partes são apenas mencionadas~(principalmente em \textbf{Resultados gerais}) e precisariam ser estudadas mais a fundo caso venham a ser incluídas na dissertação. Para incluir na dissertação a descrição da estrutura union-find retroativa, seria preciso descrever Link-Cut Trees~\cite{SleatorT1981}, uma ED poderosa mas complicada, que precisaria de um capítulo só seu, ou mais, já que ela usa splay trees~\cite{SleatorT1985}, que também poderiam ser descritas na dissertação.

A princípio, pretendo descrever na dissertação as versões retroativas de pilhas, deques, filas de prioridade e union-find. Este último tópico, como requer explicação de outras EDs complicadas, pode deixar de ser incluído se os outros tópicos tomarem muito tempo.

\subsection{Preliminares}

Alguns tópicos mais simples precisam ser explicados no início da tese. Algoritmos para os problemas
Ancestral Comum Mais Profundo e Ancestral de Nível~\cite{BenderM-F2004, Myers83}, % ficou feia a traducao
por exemplo, são usados na parte de persistência como caixa preta, e ainda precisam ser explicados. Eu conheço estes problemas bem, e já estudei e implementei soluções pra eles diversas vezes, então escrever esta parte não deve demorar.

Se apresentarmos a implementação de pilhas e deques retroativas, como planejamos, pode ser interessante escrever um capítulo sobre~ABBs modificadas, detalhando a implementação das ABBs usadas em filas, pilhas e deques retroativas. Cormen et al.~\cite{CormenAugment} descreve como modificar ABBs para armazenar valores adicionais.

\subsection{Cronograma}

\newcommand{\n}[1]{\nref{#1}}
\newcommand{\s}{\surd}

\begin{center}
\noindent
\begin{tabular}{|c|c|c|c|c|c|c|c|c|c|c|c|c|}\hline
Ativividade & Jan  & Fev  & Mar  & Abr  & Mai  & Jun  & Jul  & Ago  & Set  & Out  & Nov  & Dez  \\
\hline
\n{it:lca}  & $\s$ & $\s$ &      &      &      &      &      &      &      &      &      &      \\
\arrayrulecolor{gray!80!}\hline\arrayrulecolor{black}
\n{it:abb}  &      & $\s$ & $\s$ &      &      &      &      &      &      &      &      &      \\
\arrayrulecolor{gray!80!}\hline\arrayrulecolor{black}
\n{it:deq}  &      &      &      & $\s$ & $\s$ &      &      &      &      &      &      &      \\
\arrayrulecolor{gray!80!}\hline\arrayrulecolor{black}
\n{it:pq}   &      &      &      &      &      & $\s$ & $\s$ & $\s$ &      &      &      &      \\
\arrayrulecolor{gray!80!}\hline\arrayrulecolor{black}
\n{it:spl}  &      &      &      &      &      &      &      & $\s$ & $\s$ &      &      &      \\
\arrayrulecolor{gray!80!}\hline\arrayrulecolor{black}
\n{it:link} &      &      &      &      &      &      &      &      & $\s$ & $\s$ &      &      \\
\arrayrulecolor{gray!80!}\hline\arrayrulecolor{black}
\n{it:uf}   &      &      &      &      &      &      &      &      &      &      & $\s$ &      \\
\arrayrulecolor{gray!80!}\hline\arrayrulecolor{black}
\n{it:tese} &      &      &      &      &      &      &      &      &      & $\s$ & $\s$ & $\s$ \\
\hline
\end{tabular}
\end{center}

\vspace{2mm}

\noindent 
{\bf Legenda:}
\begin{enumerate}
\item Estudo, implementação e escrita de um capítulo sobre Ancestral Comum Mais Profundo e Ancestral de Nível, usando duas técnicas~\cite{BenderM-F2004, Myers83}. \label{it:lca}
\item Estudo, implementação e escrita de um capítulo sobre ABBs modificadas, mostrando as técnicas necessárias para implementar filas, pilhas e deques retroativas~\cite{CormenAugment}. \label{it:abb}
\item Estudo, implementação e escrita de um capítulo sobre pilhas e deques retroativas~\cite{DemaineIL2007}. \label{it:deq}
\item Estudo, implementação e escrita de um capítulo sobre filas de prioridades parcialmente retroativas~\cite{DemaineIL2007}. \label{it:pq}
\item Estudo, implementação e escrita de um capítulo sobre Splay Trees~\cite{SleatorT1985}. \label{it:spl}
\item Estudo, implementação e escrita de um capítulo sobre Link Cut Trees~\cite{SleatorT1981}. \label{it:link}
\item Estudo, implementação e escrita de um capítulo sobre union-find retroativo~\cite{DemaineIL2007}. \label{it:uf}

\item Finalização da escrita da dissertação. \label{it:tese}
\end{enumerate}

\end{document}
