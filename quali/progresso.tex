\documentclass[quali.tex]{subfile}

\begin{document}

\section{Andamento do trabalho} \label{sec:progresso}

Além das estruturas apresentadas nesse documento, detalho a seguir o meu progresso na dissertação de mestrado. Além de preparar o texto da dissertação, tenho implementado cada uma das EDs estudadas. Estas estruturas foram implementadas em~C\texttt{++}, de forma mais próxima possível ao pseudocódigo, devidamente testadas e documentadas. A documentação fica em~\mbox{\href{https://yancouto.github.io/mestrado/}{\texttt{yancouto.github.io/mestrado}}}. O código fonte do texto da dissertação e as implementações mais atualizadas estão no~GitHub (\href{https://github.com/yancouto/mestrado}{\texttt{github.com/yancouto/mestrado}}). Uma cópia atualizada da dissertação fica em\mbox{~\href{https://yancouto.github.io/mestrado/thesis.pdf}{\texttt{yancouto.github.io/mestrado/thesis.pdf}}}.



De forma mais informal, também disponibilizo problemas (no estilo ACM-ICPC) sobre alguns tópicos vistos durante o mestrado em~\mbox{\href{http://codeforces.com/gym/101261}{\texttt{codeforces.gym/101261}}}.

\subsection{Persistência}

No tópico de persistência, o texto já contém todo o material que pretendemos abordar e está, em princípio, completo (a menos de revisões). Os capítulos desta parte são:

\subsubsection*{Pilhas e filas}
	Apresenta a descrição destas EDs como neste documento.
\subsubsection*{Deque com Level Ancestor e Lowest Common Ancestor}
	Apresenta uma implementação de deques persistente de minha autoria, que utiliza algoritmos de Level Ancestor, e é similar a implementação de pilhas e filas persistentes.
\subsubsection*{Deque recursiva}
	Apresenta uma implementação de deque persistente de Kaplan~\cite{Kaplan2001}, usando uma estrutura recursiva que se inspira em contadores binários.
\subsubsection*{Deque de Kaplan e Tarjan}
	Apresenta uma implementação de deque persistente, inicialmente apresentada por Kaplan e Tarjan~\cite{KaplanT1999}, se baseia na do capítulo anterior, e consome tempo constante em todas as operações exceto~\textsc{k-th}.
\subsubsection*{Técnicas gerais}
	Define mais formalmente uma estrutura de dados, e detalha quatro técnicas, entre elas algumas dadas por~Driscoll et al.~\cite{DriscollSST1989}, para deixar qualquer ED, sob algumas suposições, persistente.
\subsubsection*{Árvore rubro-negra}
	Apresenta a implementação de uma árvore rubro-negra parcialmente persistente, utilizando um dos métodos descritos no capítulo anterior, aplicado sobre a implementação de árvore rubro-negra descrita por~Cormen et al.~\cite{CormenRedBlack}.

\subsection{Retroatividade}

Esta parte ainda está em desenvolvimento. Por hora, apenas o capítulo sobre filas (presente nesse documento) foi escrito. A fila retroativa foi implementada, testada e documentada.

\end{document}
