\errorcontextlines=200
\documentclass[10pt, compress]{beamer}
\usetheme{m}

\usepackage[portuguese]{babel}
\usepackage{../thesis/code}
\usepackage{booktabs}
\usepackage[scale=2]{ccicons}
\usepackage{minted}
\newcommand{\V}[1]{\mathit{#1}}
\algrenewcommand\algorithmicindent{1em}
\usemintedstyle{trac}
\newcommand{\Oh}{\mathcal{O}}

\usepackage{tikz}
\usepackage{tikz-qtree}
\usetikzlibrary{matrix,backgrounds, decorations.pathreplacing, automata, arrows, positioning}
\tikzset{ edge from parent path = {(\tikzparentnode) -- (\tikzchildnode)}}

\tikzset{onslide/.code args={<#1>#2}{%
	\only<#1>{\pgfkeysalso{#2}} % \pgfkeysalso doesn't change the path
}}
\tikzset{
	invisible/.style={opacity=0},
	visible on/.style={alt={#1{}{invisible}}},
	alt/.code args={<#1>#2#3}{%
		\alt<#1>{\pgfkeysalso{#2}}{\pgfkeysalso{#3}} % \pgfkeysalso doesn't change the path
	},
	blink/.style={onslide={<#1> mLightBrown}},
	appear/.style={visible on=<#1->, blink=#1}
}


\title{Estruturas de dados persistentes}
\subtitle{Yan Soares Couto}
\date{2019}
\author{Orientadora: Cristina Gomes Fernandes}
\institute{Instituto de Matemática e Estatística}

\begin{document}

\maketitle

\begin{frame}[fragile]
	\frametitle{Persistência}

	\vfill

	\begin{center}
	\begin{minipage}{0.2\linewidth}
	\begin{tikzpicture}[sibling distance=15pt]
		\Tree [.0
			[.1 [.2 3 [.4 5 ] ] ]
			6
		]
	\end{tikzpicture}
	\end{minipage}
	\begin{minipage}{0.78\linewidth}
	\begin{table}
	\centering
	\begin{subalgorithm}{0.5\linewidth}
		\begin{algorithmic}
			\State $p_0 =$ \func{Stack}{}
			\State $p_1 =$ \func{Push}{p_0, 5}
			\State $p_2 =$ \func{Push}{p_1, 7}
			\State $p_3 =$ \func{Push}{p_2, 6}
			\State $p_4 =$ \func{Pop}{p_2}
			\State \func{Top}{p_3}
			\State $p_5 =$ \func{Push}{p_4, 9}
			\State \func{Top}{p_4}
			\State $p_6 =$ \func{Push}{p_0, 5}
		\end{algorithmic}
	\end{subalgorithm}
	\begin{subalgorithm}{0.27\linewidth}
		\begin{algorithmic}
			\State $p_0:$
			\State $p_1:$ 5
			\State $p_2:$ 5 7
			\State $p_3:$ 5 7 6
			\State $p_4:$ 5
			\State Devolve 6
			\State $p_5:$ 5 9
			\State Devolve 5
			\State $p_6:$ 5
		\end{algorithmic}
	\end{subalgorithm}
	\end{table}
	\end{minipage}
	\end{center}
\end{frame}

\begin{frame}[fragile]
	\frametitle{Árvores -- Implementação Funcional}
	\vfill

	\tikzset{
	nodes={circle, minimum size=6mm, draw},
	n/.style = {draw=none},
	b/.style = {fill=gray!40},
	be/.style = {color=gray},
	edge from parent/.append style={->, shorten >= 2pt, shorten <= 2pt},
	sibling distance=20pt,
	level distance=25pt,
	}
	\begin{figure}
		\centering
		\begin{tikzpicture}
			\Tree [.\node(a){$a$}; [.\node(b){$b$}; \node(d){$d$}; \edge[n];\node[n]{}; ] [.\node(c){$c$}; \node(e){$e$}; \edge[n];\node[n]{}; ] ]
		\end{tikzpicture}
		\begin{tikzpicture}
			\Tree [.\node(a){$a$}; [.\node(b){$b$}; \node(d){$d$}; \edge[n];\node[n]{}; ] [.\node(c){$c$}; \node(e){$e$}; \edge[n];\node[n]{}; ] ]
			\node[above of = a, b](al){$a'$};
			\draw[->, be] (al) -- (b);
			\draw[->, be] (al) -- (c);
		\end{tikzpicture}
		\begin{tikzpicture}
			\Tree [.\node(a){$a$}; [.\node(b){$b$}; \node(d){$d$}; \edge[n];\node[n]{}; ] [.\node(c){$c$}; \node(e){$e$}; \edge[n];\node[n]{}; ] ]
			\node[above of = a, b](al){$a'$};
			\node[above of = b, b](bl){$b'$};
			\draw[->, be] (al) -- (bl);
			\draw[->, be] (al) -- (c);
			\draw[->, be] (bl) to[out=230,in=90] (d);
		\end{tikzpicture}
		\begin{tikzpicture}
			\Tree [.\node(a){$a$}; [.\node(b){$b$}; \node(d){$d$}; \edge[n];\node[b](f){$f$}; ] [.\node(c){$c$}; \node(e){$e$}; \edge[n];\node[n]{}; ] ]
			\node[above of = a, b](al){$a'$};
			\node[above of = b, b](bl){$b'$};
			\draw[->, be] (al) -- (bl);
			\draw[->, be] (al) -- (c);
			\draw[->, be] (bl) to[out=230,in=90] (d);
			\draw[->, be] (bl) to[out=320,in=90] (f);
		\end{tikzpicture}
	\end{figure}
\end{frame}

\begin{frame}[fragile]
	\frametitle{Representações Numéricas}

	\vfill
	\begin{center}
		\begin{tabular}{r l l}
			$59 = $ & 59 & (decimal) \\
			$=$ & 111011 & (binária) \\
			$=$ & 11120  & (skew-binary) \\
			$=$ & 102011 & (binária redundante regular)
		\end{tabular}
	\end{center}
\end{frame}


\plain{}{\alert{Ancestrais em árvores}}

\begin{frame}[fragile]
	\frametitle{Representação binária}

	$$43 = 101011 = 32 + 8 + 2 + 1$$

	\vfill

	\begin{figure}
		\centering
		\begin{tikzpicture}[nodes={draw, circle, minimum size=5mm}, sibling distance=25pt, edge from parent/.append style={<-, shorten <= 2pt, shorten >= 2pt}, level distance=35pt, scale=1]
			\Tree [.\node(v1){}; [.\node(v2){}; [.\node(v3){}; [.\node(v4){}; [.\node(v5){}; ] ] [.\node(v6){}; [.\node(v7){}; ] ] ] [.\node(v8){}; ] [.\node(v9){}; ] ] [.\node(v10){}; [.\node(v11){}; ] ] ]

			\tikzstyle{p2}=[->, shorten >= 2pt, shorten <= 2pt, dashed, >=stealth]
			\tikzstyle{p4}=[->, shorten >= 2pt, shorten <= 2pt, dotted, >=stealth]

			\draw[p2] (v3) edge[out=90,in=180] (v1);
			\draw[p2] (v4) edge[out=10,in=200] (v2);
			\draw[p2] (v5) edge[out=130,in=190] (v3);
			\draw[p4] (v5) edge[out=150,in=160] (v1);
			\draw[p2] (v8) edge[out=70,in=250] (v1);
			\draw[p2] (v9) edge[out=80,in=-70] (v1);
			\draw[p2] (v11) edge[out=30,in=20] (v1);
			\draw[p4] (v7) edge[out=0,in=-10] (v1);
			\draw[p2] (v6) edge[out=80,in=240] (v2);
			\draw[p2] (v7) edge[out=50,in=-15] (v3);
		\end{tikzpicture}
	\end{figure}

\end{frame}

\newcommand{\CSB}{\textit{CSB}}
\newcommand{\J}{\operatorname{J}}

\begin{frame}[fragile]
	\frametitle{Representação skew-binary}

	\begin{figure}
	\begin{tikzpicture}[scale=.65, font=\scriptsize]
		%\draw (0, 1/10) node{x\ \ \ \ CSB\ \ };
		\draw (0, 3/4) node{\ \ $x$\ \ };
		\draw (0, 0) node{\ $\CSB$\ };
		%\draw (0, 0) grid (20, 10);
		\draw (-.5, 1/3) -- (16.5, 1/3);
		\foreach \i/\x/\csb in {1/59/11120, 2/56/11110, 3/53/11100, 4/46/11000, 5/45/10200, 6/44/10120, 7/41/10110, 8/40/10102}
		{
			\draw (\i+\i, 3/4) node{\ \x\ \ };
			\draw (\i+\i, 0) node(c\i){\csb};
		}
		\foreach \i/\j in {1/2, 2/3, 3/4, 6/7}
		{
			\draw (c\i.south) edge[bend right,->, shorten >=3pt, shorten <=3pt] node[below]{${\scriptstyle x = \J(x)}$} (c\j.south);
		}
		\foreach \i/\j in {4/5, 5/6, 7/8}
		{
			\draw (c\i.south) edge[bend right,->, shorten >=3pt, shorten <=3pt] node[below]{${\scriptstyle x = x - 1}$} (c\j.south);
		}
	\end{tikzpicture}
	\end{figure}

	\vfill

	\begin{figure}
	\begin{tikzpicture}[nodes={draw, circle, minimum size=5mm}, sibling distance=25pt, edge from parent/.append style={<-, shorten <= 2pt, shorten >= 2pt}, level distance=35pt, scale=.8]
		\Tree [.\node(v1){}; [.\node(v2){}; [.\node(v3){}; [.\node(v4){}; [.\node(v5){}; ] ] [.\node(v6){}; [.\node(v7){}; ] ] ] [.\node(v8){}; ] [.\node(v9){}; ] ] [.\node(v10){}; [.\node(v11){}; ] ] ]

		\tikzstyle{p2}=[->, shorten >= 2pt, shorten <= 2pt, dashed, >=stealth]
		\tikzstyle{p4}=[->, shorten >= 2pt, shorten <= 2pt, dotted, >=stealth]

		\draw[p2] (v2) edge[bend left] (v1);
		\draw[p2] (v4) edge[in=170,out=90] (v1);
		\draw[p2] (v10) edge[bend right] (v1);
		\draw[p2] (v11) edge[bend right] (v10);
		\draw[p2] (v9) edge[bend right] (v2);
		\draw[p2] (v3) edge[bend left] (v2);
		\draw[p2] (v8) edge[bend left] (v2);
		\draw[p2] (v5) edge[bend left] (v4);
		\draw[p2] (v7) edge[bend right] (v6);
		\draw[p2] (v6) edge[in=-90,out=10] (v1);

	\end{tikzpicture}
	\end{figure}


\end{frame}


\plain{}{\alert{Estruturas}}

\begin{frame}[fragile]
	\frametitle{Pilha}

	\vfill
	\begin{figure}
		\centering
		\begin{tikzpicture}[sibling distance=25pt, nodes={draw, minimum size=6mm}, edge from parent/.append style={<-, shorten <= 5pt, shorten >= 5pt}, level distance=40pt]
			\Tree [.\node[minimum size=6mm](root){};
				\edge[orange]; [.\node[orange](p14){5}; \edge[orange]; [.\node[orange](p2){7}; [.\node(p3){6}; ] ] \node(p5){9}; ]
				\node(p6){5};
			]
			\draw (root.south west) edge (root.north east);
			\draw[->, shorten >= 10pt] node[draw=none, left = 1cm of p2]{$p$} edge (p2);

		\end{tikzpicture}
	\end{figure}
\end{frame}

\begin{frame}[fragile]
	\frametitle{Fila}

	\vfill
	\begin{figure}
		\centering
		\begin{tikzpicture}[sibling distance=25pt, nodes={draw, minimum size=6mm}, edge from parent/.append style={<-, shorten <= 5pt, shorten >= 5pt}, level distance=40pt]
			\Tree [.\node[minimum size=6mm](root){};
				[.\node[blink=1](p14){5}; \edge[blink=1]; [.\node[orange](p2){7}; \edge[orange]; [.\node[orange](p3){6}; ] ] \node(p5){9}; ]
				\node(p6){5};
			]
			\draw (root.south west) edge (root.north east);
			\draw[->, shorten >= 10pt] node[draw=none, left = 1cm of p3]{$\V{lst}$} edge (p3);
			\only<1>{\draw[->, shorten >= 10pt] node[draw=none, left = 1cm of p14]{$\V{fst}$} edge (p14);}
			\only<2>{\draw[->, shorten >= 10pt] node[draw=none, left = 1cm of p2]{$\V{fst}$} edge (p2);}

		\end{tikzpicture}
	\end{figure}
\end{frame}

\begin{frame}[fragile]
	\frametitle{Deque com LA e LCA}

	\vfill
	\begin{figure}
		\centering
		\begin{tikzpicture}[sibling distance=30pt, nodes={draw, minimum size=6mm}, edge from parent/.append style={<-, shorten <= 5pt, shorten >= 5pt}, level distance=40pt, level 3/.style={sibling distance=80pt}, level 1/.style={sibling distance=80pt}, scale=0.8]

			\Tree [.\node[minimum size=6mm](root){};
			[.\node[orange](v3){3}; \edge[orange]; [.\node[orange](v2){2}; \edge[orange, visible on=<2->]; \node[visible on=<2->, orange](v1){1}; [.\node(v9){9}; ] ] \edge[alt={<1-2>{orange}{}}]; \node[alt={<1-2>{orange}{}}](v4){4}; ]
			\node(v6){6};
			]

			\draw (root.south west) edge (root.north east);

			%\draw[->, shorten >= 10pt] node[draw=none, left = 1cm of v6]{$\V{first}_9$} edge (v6);
			%\draw[->, shorten >= 10pt] node[draw=none, right = 1cm of v6]{$\V{last}_9$} edge (v6);

			\only<1>{\draw[->, shorten >= 10pt] node[draw=none, left = 1cm of v2]{$\V{fst}$} edge (v2);}
			\only<2->{\draw[->, shorten >= 10pt] node[draw=none, left = 1cm of v1]{$\V{fst}$} edge (v1);}

			\only<1-2>{\draw[->, shorten >= 10pt] node[draw=none, right = 1cm of v4]{$\V{lst}$} edge (v4);}
			\only<3>{\draw[->, shorten >= 10pt] node[draw=none, right = 1cm of v3]{$\V{lst}$} edge (v3);}

		\end{tikzpicture}
	\end{figure}
\end{frame}

\begin{frame}[fragile]
	\frametitle{Deque recursiva}

	\tikzset{drawnode1/.style={row 2 column #1/.style={nodes={draw}}}}
	\tikzset{drawnode2/.style={row 3 column #1/.style={nodes={draw}}}}
	\tikzset{drawnode3/.style={row 4 column #1/.style={nodes={draw}}}}

	\vfill
	\begin{figure}
	\centering
	\begin{tikzpicture}[
	array/.style={matrix of nodes,nodes={draw=none, minimum size=4mm, anchor=center}, nodes in empty cells, row sep=1cm},
	drawnode1/.list={9, 17},
	drawnode2/.list={2, 3, 9, 15, 16},
	drawnode3/.list={4, 5, 6, 7, 9, 11, 12, 13, 14},
	row 2 column 1/.style={nodes={draw, blink=2, blink=3}},
	row 3 column 2/.style={nodes={draw, blink=3}},
	row 3 column 3/.style={nodes={draw, blink=3}},
	row 2/.style={row sep=0cm}
	]

	\matrix[array] (array) {
	$\V{prefix}$ &  &  &  & & & & & $\V{center}$ &  & & & & &  &  & $\V{suffix}$ \\[-1cm]
	\alt<3>{4}{\only<1>{2}} &  &  &  & & & & &  &  & & & & &  &  &  \\
	& \alt<3>{5 & 6}{ & }  &  & & & & &  & & & & &  & \only<1-3>{7} & 8 &  \\
	\only<1-2>{& & & 3 & 4 & 5 & 6 &  &  &  &  &  &  &  & & &} \\
	};
	
	\draw[dashed] (array-2-9.south west) -- (array-3-2.north west);
	\draw[dashed] (array-2-9.south east) -- (array-3-16.north east);
	\only<1-2>{\draw[dashed] (array-3-9.south west) -- (array-4-4.north west);}
	\only<1-2>{\draw[dashed] (array-3-9.south east) -- (array-4-14.north east);}

	\path[draw] (array-2-17.south west) -- (array-2-17.north east);
	\only<1-2>{\path[draw] (array-3-2.south west) -- (array-3-3.north east);}
	\only<3>{\path[draw] (array-3-9.south west) -- (array-3-9.north east);}
	\only<1-2>{\path[draw] (array-4-9.south west) -- (array-4-9.north east);}
	\only<1-2>{\path[draw] (array-4-11.south west) -- (array-4-14.north east);}

	\only<2>{\path[draw, orange] (array-2-1.south west) -- (array-2-1.north east);}
	
	\node[left of=array-2-1, xshift=-.2cm]{Nível 1};
	\node[left of=array-3-1, xshift=-.2cm]{Nível 2};
	\only<1-2>{\node[left of=array-4-1, xshift=-.2cm]{Nível 3};}

	
	\end{tikzpicture}
	\end{figure}
\end{frame}

\begin{frame}[fragile]
	\frametitle{Deque de Kaplan e Tarjan}

	\tikzset{drawnode1/.style={row 1 column #1/.style={nodes={draw}}}}
	\tikzset{drawnode2/.style={row 2 column #1/.style={nodes={draw}}}}
	\tikzset{drawnode3/.style={row 3 column #1/.style={nodes={draw}}}}

	\begin{figure}[h]
	\centering
	\begin{tikzpicture}[
	array/.style={matrix of nodes,nodes={draw=none, text centered, text width=1em, minimum height=2.5ex, font=\small, align=center, anchor=center}, nodes in empty cells, row sep=1cm},
	drawnode1/.list={1, 2, 3, 4, 5, 7, 9, 10, 11, 12, 13},
	drawnode2/.list={1, 2, 3, 4, 5, 7, 9, 10, 11, 12, 13},
	drawnode3/.list={1, 2, 3, 4, 5, 7, 9, 10, 11, 12, 13},
	row 2/.style={nodes={minimum height=6ex}},
	row 3/.style={nodes={minimum height=10.7ex}},
	]

	\matrix[array] (array) {
	0 & 1 &  &   &          & &     & &  &  &  & & \\
	{2 \\ 3} & & & & &  & &      & {8 \\ 9} & {10 \\ 11} & & & \\
	{4 \\ 5 \\ 6 \\ 7} & & & & &  & &      & & & & & \\
	};
	
	\draw[dashed] (array-1-7.south west) -- (array-2-1.north west);
	\draw[dashed] (array-1-7.south east) -- (array-2-13.north east);
	\draw[dashed] (array-2-7.south west) -- (array-3-1.north west);
	\draw[dashed] (array-2-7.south east) -- (array-3-13.north east);

	%\draw (root.south west) edge (root.north east);
	\newcommand{\dnull}[1] { \path[draw] (#1.south west) -- (#1.north east); }

	\foreach \p in {1-3, 1-4, 1-5, 1-9, 1-10, 1-11, 1-12, 1-13, 2-2, 2-3, 2-4, 2-5, 2-11, 2-12, 2-13, 3-2, 3-3, 3-4, 3-5, 3-9, 3-10, 3-11, 3-12, 3-13, 3-7} {\dnull{array-\p}}

	% \path[draw] (array-2-17.south west) -- (array-2-17.north east);
	% \path[draw] (array-3-2.south west) -- (array-3-3.north east);
	% \path[draw] (array-4-9.south west) -- (array-4-9.north east);
	% \path[draw] (array-4-11.south west) -- (array-4-14.north east);
	
	\node[left= 3pt of array-1-1, xshift=-.2cm]{\small Nível 1};
	\node[left= 3pt of array-2-1, xshift=-.2cm]{\small Nível 2};
	\node[left= 3pt of array-3-1, xshift=-.2cm]{\small Nível 3};

	\node[above=.5cm of array-1-3]{$\V{prefix}$};
	\node[above=.5cm of array-1-7]{$\V{center}$};
	\node[above=.5cm of array-1-11]{$\V{suffix}$};

	%\node[above = 3pt of array-1-3]{Dígito~0};
	%\node[above = 3pt of array-1-11]{Dígito~2};

	%\node[above = 3pt of array-2-3]{Dígito~1};
	%\node[above = 3pt of array-2-11]{Dígito~0};

	%\node[above = 3pt of array-3-3]{Dígito~1};
	%\node[above = 3pt of array-3-11]{Dígito~2};
	
	\end{tikzpicture}
	\end{figure}
\end{frame}

\begin{frame}[fragile]
	\frametitle{Árvore rubro-negra}
	Implementação funcional: Totalmente persistente
	
	$\Oh(n \lg n)$ tempo e espaço.
	\vfill
	Implementação node copying: Parcialmente persistente,
	
	$\Oh(n \lg n)$ tempo e~$\Oh(n)$ espaço.
	\vfill

	\begin{tikzpicture}[
	ed/.style = {densely dashed, shorten >= 5pt},
	nodes = {draw, circle, minimum size = 8mm},
	edge from parent path = {(\tikzparentnode) -- (\tikzchildnode)},
	sibling distance=20pt,
	n/.style = {draw=none},
	r/.style = {fill=white},
	b/.style = {fill=gray},
	edge from parent/.append style={-, shorten >= 0, shorten <= 0}
	]
	\Tree [.\node[b]{7}; [.\node[b]{5}; [.\node[r]{1}; \edge[ed];\node[n]{}; \edge[ed];\node[n]{}; ] \edge[ed];[.\node[n]{}; ] ] [.\node[b]{10}; [.\node[r]{8}; \edge[ed];\node[n]{}; \edge[ed];\node[n]{}; ] \edge[ed];[.\node[n]{}; ] ] ]
	\end{tikzpicture}
\end{frame}

\begin{frame}[fragile]
	\frametitle{Localização de ponto}

	\only<1>{
	\begin{figure}
	\centering
	\begin{tikzpicture}[]

	\newcounter{size}
	\newcommand\polyg[2]{
		\setcounter{size}{0}
		\foreach \point [count=\i] in #1 {
			\stepcounter{size}
			\node[coordinate] (p1-\i) at \point {};
		}
		\fill[#2, opacity=0.3] (p1-1) \foreach \i in {2, ..., \value{size}} {-- (p1-\i)} -- cycle;
		\draw[#2, opacity=1] (p1-1) \foreach \i in {2, ..., \value{size}} {-- (p1-\i)} -- cycle;
	}

	%\draw[opacity=0.2, dashed] (-8, 0) grid (2, 6);

	\polyg {{(-7, 4), (-5, 5), (-3, 4), (-5, 1), (-7, 1)}}{blue};
	\polyg {{(-2, 5), (-1, 4), (-1, 1), (0, 1), (1, 5)}}{violet};
	\polyg {{(-2, 4), (-4, 1), (-2, 1)}}{brown};

	\newcommand\drawpts[2]{ \foreach \point in #1 { \draw[fill=#2, color=#2] \point circle (0.075); }}

	\drawpts{{(-5, 3)}}{blue};
	\drawpts{{(-2.5, 2)}}{brown};
	\drawpts{{(0, 4), (0, 2)}}{violet};
	\drawpts{{(-3, 5), (-7, 5), (-2, 0)}}{black};

	\draw[dashed, black] (-3, 5) -- (-3, 4);
	\draw[dashed, black] (-7, 5) -- (-7, 4);
	\draw[dashed, blue]  (-5, 3) -- (-5, 1);
	\draw[dashed, brown]  (-2.5, 2) -- (-2.5, 1);
	\draw[dashed, violet]  (0, 4) -- (0, 2);
	\draw[dashed, violet]  (0, 2) -- (0, 1);

	\end{tikzpicture}
	\end{figure}}

	\only<2>{
	\begin{figure}
	\centering
	\begin{tikzpicture}[]

	\newcommand\polyg[2]{
		\setcounter{size}{0}
		\foreach \point [count=\i] in #1 {
			\stepcounter{size}
			\node[coordinate] (p1-\i) at \point {};
		}
		\fill[#2, opacity=0.3] (p1-1) \foreach \i in {2, ..., \value{size}} {-- (p1-\i)} -- cycle;
		\draw[#2, opacity=1] (p1-1) \foreach \i in {2, ..., \value{size}} {-- (p1-\i)} -- cycle;
	}

	%\draw[opacity=0.2, dashed] (-8, 0) grid (2, 6);

	\polyg {{(-7, 4), (-5, 5), (-3, 4), (-5, 1), (-7, 1)}}{blue};
	\polyg {{(-2, 5), (-1, 4), (-1, 1), (0, 1), (1, 5)}}{violet};
	\polyg {{(-2, 4), (-4, 1), (-2, 1)}}{brown};

	\draw[dashed] (-7, 0) -- (-7, 6);
	\draw[dashed] (-5, 0) -- (-5, 6);
	\draw[dashed] (-4, 0) -- (-4, 6);
	\draw[dashed] (-3, 0) -- (-3, 6);
	\draw[dashed] (-2, 0) -- (-2, 6);
	\draw[dashed] (-1, 0) -- (-1, 6);
	\draw[dashed] (0, 0) -- (0, 6);
	\draw[dashed] (1, 0) -- (1, 6);

	\end{tikzpicture}
	\end{figure}
	}
\end{frame}

\plain{}{\alert{Obrigado!}}

\end{document}
