\documentclass[main.tex]{subfiles}

\begin{document}

\chapter{Deque Recursiva} \label{cap:deque2_persist}

Neste capítulo, apresentaremos um implementação persistente desta estrutura. A implementação é discutida por Kaplan~\cite{Kaplan2001} e serve como base para a implementação discutida no Capítulo~\ref{cap:deque3_persist}.

%Diferente das definições do Capítulo~\ref{cap:deque1_persist}, usaremos que a função~\textsc{Pop} devolve o elemento removido, assim como a nova deque. Isso é usado para facilitar a implementação da função~\textsc{Pop}.
%\begin{itemize}
%	\item \func{PopFront}{d} --- Retorna o primeiro elemento de~$d$, e uma deque igual a~$d$, mas sem este elemento.
%	\item \func{PopFront}{d} --- Retorna o último elemento de~$d$, e uma deque igual a~$d$, mas sem este elemento.
%\end{itemize}

\section{Representação}

A implementação usual de uma deque usando lista ligada utiliza uma lista duplamente ligada, e então para adicionar um elemento é necessário \emph{mudar} campos de nós já existentes da lista, o que não resulta numa implementação persistente como ocorreu com a pilha. Uma solução é fazer como na implementação do Capítulo~\ref{cap:deque1_persist}, que transforma esta lista em uma arborescência. Neste capítulo, trataremos isto de outra forma.

Para conseguirmos manter a implementação funcional, mudaremos como uma deque é representada, com um esquema de representação recursiva. Para facilitar a descrição, é necessário deixar claro o conjunto de elementos que a deque pode armazenar. Seja~$T$ esse conjunto, ou seja, operações de~\API{Push} recebem elementos de~$T$ como argumento e operações de~\API{Pop} devolvem elementos desse conjunto.

Uma deque persistente que armazena elementos do conjunto~$T$ é um nó com três campos opcionais:~$\V{prefix}, \V{center}$, e $\V{suffix}$; além de um campo obrigatório~$\V{size}$. Um campo ser opcional significa que ele pode não estar presente. Quando não estiver presente, o campo assume o valor~\keyword{null}. Os campos~$\V{prefix}$ e~$\V{suffix}$ armazenam valores de~$T$, o campo~$\V{center}$ armazena uma deque persistente que armazena elementos do conjunto~$T \times T$, ou seja, pares ordenados de~$T$, e o campo~$\V{size}$ armazena o tamanho da deque.

%Explicação alternativa
%Seja~$d$ uma deque persistente que armazena elementos do tipo~$T$. Se $d = \keyword{null}$, então $d$ é uma deque vazia. Caso contrário,~$d$ é uma tripla~$(\V{prefix}(d), \V{center}(d), \V{suffix}(d))$, onde~$\V{preffix}(d)$ e~$\V{suffix}(d)$ são elementos de~$T$ ou~\keyword{null}, e~$\V{center}(d)$ é uma deque persistente que armazena elementos de~$T \times T$, ou seja, pares ordenados de~$T$.

\tikzset{drawnode1/.style={row 2 column #1/.style={nodes={draw}}}}
\tikzset{drawnode2/.style={row 3 column #1/.style={nodes={draw}}}}
\tikzset{drawnode3/.style={row 4 column #1/.style={nodes={draw}}}}

\begin{figure}
\centering
\begin{tikzpicture}[
array/.style={matrix of nodes,nodes={draw=none, minimum size=7mm, anchor=center}, nodes in empty cells, row sep=2cm},
drawnode1/.list={1, 9, 17},
drawnode2/.list={2, 3, 9, 15, 16},
drawnode3/.list={4, 5, 6, 7, 9, 11, 12, 13, 14},
row 2/.style={row sep=0cm}
]

\matrix[array] (array) {
$\V{prefix}$ &  &  &  & & & & & $\V{center}$ &  & & & & &  &  & $\V{suffix}$ \\[-1.7cm]
2 &  &  &  & & & & &  &  & & & & &  &  &  \\
 &  &  &  & & & & &  & & & & &  & 7 & 8 &  \\
& & & 3 & 4 & 5 & 6 &  &  &  &  &  &  &  & & & \\
 };
 
 \draw[dashed] (array-2-9.south west) -- (array-3-2.north west);
 \draw[dashed] (array-2-9.south east) -- (array-3-16.north east);
 \draw[dashed] (array-3-9.south west) -- (array-4-4.north west);
 \draw[dashed] (array-3-9.south east) -- (array-4-14.north east);

%\draw (root.south west) edge (root.north east);
 \newcommand{\dnull}[1] { \path[draw] (#1.south west) -- (#1.north east); }

 \foreach \p in {2-17, 3-2, 3-3, 4-9, 4-11, 4-12, 4-13, 4-14} {\dnull{array-\p}}
 
 \node[left of=array-2-1, xshift=-.2cm]{Nível 1};
 \node[left of=array-3-1, xshift=-.2cm]{Nível 2};
 \node[left of=array-4-1, xshift=-.2cm]{Nível 3};

 
\end{tikzpicture}
\caption{Deque persistente} \label{fig:deque_ex}
\end{figure}

Como indicado pelos nomes,~$\V{prefix}$ armazena um prefixo da sequência de elementos que é a deque,~$\V{suffix}$ representa um sufixo, e~$\V{center}$ os elementos do meio. A Figura~\ref{fig:deque_ex} mostra uma possível deque persistente, cuja sequência é~$(2,3,4,5,6,7,8)$. Note que os campos~\keyword{null} são ignorados.

Observe que, para uma deque não vazia, considerando que nenhum campo~$\V{center}$ guarda uma deque vazia, o número máximo de níveis nessa estrutura é~$\lg n$, onde~$n$ é o número de elementos na deque, já que no~$i$-ésimo nível cada elemento~$\V{prefix}$ e~$\V{suffix}$ armazena~$2^{i-1}$ elementos.


\section{Operações de acesso}


\begin{algorithm}
\caption{Operações de acesso para uma deque persistente.} \label{lst:deque_acesso}
\begin{algorithmic}[1]

\Function{\API{Deque}}{\null}
	\State \Return \Null
\EndFunction

\Function{\API{Front}}{$d$}
	\If{$d.\V{prefix} \neq \Null$}
		\State \Return $d.\V{prefix}$
	\ElsIf{$d.\V{center} = \Null$}
		\State \Return $d.\V{suffix}$
	\Else
		\State \Return \funcAPI{Front}{d.\V{center}}$.\V{first}$
	\EndIf
\EndFunction

\Function{\API{Back}}{$d$}
	\If{$d.\V{suffix} \neq \Null$}
		\State \Return $d.\V{suffix}$
	\ElsIf{$d.\V{center} = \Null$}
		\State \Return $d.\V{prefix}$
	\Else
		\State \Return \funcAPI{Back}{d.\V{center}}$.\V{second}$
	\EndIf
\EndFunction

% Não decidi se isso vai ser uma operação da deque
%\Function{\API{Size}}{$d$}
%	\If{$d = \Null$}
%		\State \Return $0$
%	\Else
%		\State \Return $d.\V{size}$
%	\EndIf
%\EndFunction


\end{algorithmic}
\end{algorithm}

Devido à forma recusiva como a estrutura foi definida, recursão é muito útil para lidar com esta estrutura.
Assumimos sempre que todas as operações chamadas são válidas, ou seja,~\funcAPI{Front}{d} e~\funcAPI{Back}{d} são chamadas apenas quando~$d$ não é vazia, e consequentemente seu valor não é~\keyword{null}. Veja o Código~\ref{lst:deque_acesso}. Para questões de implementação, os elementos de~$T \x T$ são representados como um par com campos~$\V{first}$ e~$\V{second}$, ou seja, se~$a, b \in T$ então~$(a, b) \in T \x T$ e vale que~$(a, b).\V{first} = a$ e~$(a, b).\V{second} = b$.

Vamos analisar a operação~\funcAPI{Front}{d}. Pela maneira como definimos a deque, se~$\V{prefix}$ não é nulo então ele é um prefixo de~$d$, logo este é a resposta de~\funcAPI{Front}{d}. Se não, se~$\V{center}$ não é nulo, como não guardamos deques vazias, este tem algum elemento. O valor devolvido é então o primeiro elemento de~$d.\V{center}$, e como esta é uma deque de pares de elementos do tipo~$T \times T$, temos que o primeiro elemento do primeiro par de~$d.\V{center}$ é o primeiro elemento de~$d$. Caso contrário, o único elemento da deque é~$d.\V{suffix}$ e o valor devolvido também é correto.
Portanto, a função~\funcAPI{Front}{d} funciona corretamente.

Note que~\funcAPI{Back}{d} é simétrica a~\funcAPI{Front}{d}. Basta trocar~$\V{prefix}$ com~$\V{suffix}$ e~$\V{first}$ com~$\V{second}$, o mesmo ocorrerá com as outras operações, pela simetria da estrutura. Então nas próximas seções apenas a versão~\textsc{Front} de~\API{Push} e~\API{Pop} será detalhada.

Ambas as funções apresentadas consomem tempo~$\Oh(\lg n)$, onde~$n$ é o tamanho da deque, e não modificam a estrutura.

\section{Operações de modificação}

Ao adicionar um elemento~$x$ no início da deque~$d$, se~$d.\V{prefix}$ for nulo, podemos colocar~$x$ no campo~$\V{prefix}$, e então a deque continua válida e com~$x$ no início. Se~$d.\V{prefix}$ não for nulo, podemos juntá-lo com~$x$ e então inserir o par~$(x, d.\V{prefix})$ em~$d.center$, e tornar~$d.\V{prefix}$ nulo, assim a deque continua válida e com~$x$ no início.

Analogamente, ao remover um elemento do início de~$d$, se~$d.\V{prefix}$ não for nulo, podemos remover esse elemento, caso contrário, podemos remover um elemento de~$d.\V{center}$ e colocar apenas o segundo elemento em~$d.\V{prefix}$, já que os elementos de~$d.\V{center}$ são pares, removendo assim o primeiro elemento da deque. Note que precisamos do elemento removido, então usaremos uma função auxiliar~\func{PopFrontAux}{d} que devolve um par com o primeiro elemento de~$d$ e uma cópia de~$d$ sem este elemento.

Se fizermos as operações desta forma, estaremos mudando a estrutura, que não será funcional. Note que apenas mudamos um prefixo dos níveis, e cada nível consiste de apenas um número constante de dados (se considerarmos que os pares são na verdade ponteiros para pares).
Assim, se apenas copiarmos todos os nós que iremos modificar, mantemos a estrutura funcional. Assumimos que~\mbox{\keyword{new} \func{Node}{\V{pr}, \V{ct}, \V{sf}, \V{sz}}} devolve um nó como discutido, com~$\V{prefix}$,~$\V{center}$,~$\V{suffix}$ e~$\V{size}$ inicializados com~$\V{pr}$,~$\V{ct}$,~$\V{sf}$ e~$\V{sz}$, respectivamente. Veja o Código~\ref{lst:deque_mod}.

\begin{algorithm}
\caption{Operações de modificação para uma deque.} \label{lst:deque_mod}
\begin{algorithmic}[1]

\Function{\API{PushFront}}{$d, x$}
	\If{$d = \Null$}
		\State \Return \New \func{Node}{x, \Null, \Null, 1} \label{line:dm:puf1}
	\ElsIf{$d.\V{prefix} = \Null$} \label{line:dm:puf2}
		\State \Return \New \func{Node}{x, d.\V{center}, d.\V{suffix}, d.\V{size} + 1}\label{line:dm:puf2_1}
	\Else
		\State \Return \New \func{Node}{\Null, \funcAPI{PushFront}{d.\V{center}, (x, d.\V{prefix})}, d.\V{suffix}, d.\V{size} + 1} \label{line:dm:puf3}
	\EndIf
\EndFunction

\Function{\API{PopFront}}{$d$}
	\State \Return $\Call{PopFrontAux}{d}.\V{second}$
\EndFunction


\Function{PopFrontAux}{$d$}
	\If{$d.\V{prefix} \neq \Null \AND d.\V{center} = \Null \AND d.\V{suffix} = \Null$} \label{line:dm:pof0}
		\State \Return $(d.\V{prefix}, \funcAPI{Deque}{})$
	\ElsIf{$d.\V{prefix} \neq \Null$} \label{line:dm:pof1}
		\State \Return $(d.\V{prefix}, \New \Call{Node}{\Null, d.\V{center}, d.\V{suffix}, d.\V{size} - 1})$ \label{line:dm:pof1_2}
	\ElsIf{$d.\V{center} = \Null$}
		\State \Return $(d.\V{suffix}, \funcAPI{Deque}{\null})$ \label{line:dm:pof2}
	\Else
		\State $(x, c) = \Call{PopFrontAux}{d.\V{center}}$
		\State \Return $(x.\V{first}, \New \Call{Node}{x.second, c, d.\V{suffix}, d.\V{size} - 1})$ \label{line:dm:pof3}
	\EndIf
\EndFunction

\end{algorithmic}
\end{algorithm}

\begin{proposition}
A operação~\funcAPI{PushFront}{d, x} funciona corretamente, devolvendo uma cópia de~$d$ com~$x$ inserido em seu início, sem modificar~$d$.
\end{proposition}

\begin{proof}
A prova será feita por indução no número de níveis da estrutura recursiva da deque.

A base se dá quando~$d$ é nulo, nesse caso a deque é vazia e a linha~\nref{line:dm:puf1} devolve uma deque com~$x$ em seu campo~$\V{prefix}$ e os outros campos nulos. Assuma então que a proposição vale para todas as deques com número de níveis menor que~$d$.

Se~$d.\V{prefix}$ é nulo, podemos armazenar~$x$ nessa posição e teremos adicionado~$x$ como primeiro elemento. Como não podemos modificar os campos de~$d$, criamos um novo nó com campo~$\V{prefix}$ como~$x$ e os outros campos iguais aos de~$d$. Esse caso é então tratado corretamente nas no~\keyword{if} das linhas~\nref{line:dm:puf2} e~\nref{line:dm:puf2_1}.

Caso contrário, juntamos~$x$ e~$d.\V{prefix}$ em um par e o adicionamos recursivamente no começo de~$d.\V{center}$. Como a deque~$d.\V{center}$ tem menos níveis que~$d$, a hipótese de indução vale e a operação funciona nesse caso.
\end{proof}

De maneira simétrica temos que a operação~\funcAPI{PushBack}{d, x} também está correta. Provaremos a seguir que~\func{PopFrontAux}{d} funciona corretamente, e portanto~\funcAPI{PopFront}{d} também, já que apenas devolve a deque devolvida por~\func{PopFrontAux}{d}. Consequentemente,~\funcAPI{PopBack}{d} está correta.

\begin{proposition}
A função~\func{PopFrontAux}{d} funciona corretamente, devolvendo, sem modificar~$d$, o primeiro elemento de~$d$ e uma cópia de~$d$ sem este elemento.
\end{proposition}

\begin{proof}
A prova também será feita por indução no número de níveis da estrutura da deque.

    Se~$d.\V{prefix}$ não é nulo, basta remover e devolver este elemento, assim como uma cópia da deque sem ele. Se~$d.\V{prefix}$ era o único nó da estrutura, devolve-se uma deque vazia. Este caso é tratado na linha~\nref{line:dm:pof0}. Caso contrário, o~\keyword{if} das linhas~\nref{line:dm:pof1} e~\nref{line:dm:pof1_2} devolve~$d.\V{prefix}$ e cria um novo nó que não tem esse campo, sem modificar~$d$, portanto trata corretamente este caso.

Caso contrário, se~$d.\V{center}$ é nulo, sabemos que~$d$ é uma deque com apenas um elemento, armazenado em~$d.\V{suffix}$, pois sabemos que~\func{PopFrontAux}{d} não é chamada se~$d$ é vazia. A linha~\nref{line:dm:pof2} trata corretamente esse caso, devolvendo~$d.\V{suffix}$ e uma nova deque vazia. A base da indução, quando a deque tem um nível, sempre satisfaz um destes casos, logo é tratada corretamente.

No último caso, usamos~\func{PopFrontAux}{d.\V{center}}, que está correta pois~$d.\V{center}$ tem menos níveis que~$d$, para remover recursivamente o primeiro elemento de~$d.\V{center}$. Esse elemento é um par de elementos de~$d$. Retornamos o primeiro elemento do par, que é o primeiro elemento de~$d$, e colocamos o segundo elemento no campo~$\V{prefix}$, removendo assim o primeiro elemento da deque. Também é necessário substituir~$\V{center}$ pela nova deque devolvida pela chamada recursiva. A linha~\nref{line:dm:pof3} devolve então um novo nó com os campos corretos.
\end{proof}

Ambas operações consomem tempo e espaço~$\Oh(\lg n)$, onde~$n$ é o tamanho da deque, pois apenas percorrem a estrutura dos nós, que tem altura~$\Oh(\lg n)$, e realizam operações que consomem tempo e espaço constante por nível.

\section{Acesso a outros elementos}

Para tornar a deque de acesso aleatório, resta implementar~\funcAPI{k-th}{d, k}. Veja o Código~\ref{lst:deque_kth}.

\begin{algorithm}
\caption{Implementação de~\func{k-th}{d, k}.} \label{lst:deque_kth}
\begin{algorithmic}[1]

\Function{\API{k-th}}{$d, k$}
	\If{$k = 1 \AND d.\V{prefix} \neq \Null$} 
		\State \Return $d.\V{prefix}$\label{line:kth:if1}
	\EndIf
	\If{$k = d.\V{size} \AND d.\V{suffix} \neq \Null$}
		\State \Return $d.\V{suffix}$ \label{line:kth:if2}
	\EndIf
	\If{$d.\V{prefix} \neq \Null$} 
		\State $k = k - 1$ \label{line:kth:if3}
	\EndIf
	\If{$k$ is odd}
		\State \Return $\funcAPI{k-th}{d.\V{center}, \ceil*{\frac{k}{2}}}.\V{first}$
	\Else
		\State \Return $\funcAPI{k-th}{d.\V{center}, \ceil*{\frac{k}{2}}}.\V{second}$
	\EndIf
\EndFunction

\end{algorithmic}
\end{algorithm}


\begin{proposition}
A função~\func{k-th}{d, k} devolve o~$k$-ésimo elemento de~$d$, para~$k \leq d.\V{size}$.
\end{proposition}

\begin{proof}
As linhas~\nref{line:kth:if1} e~\nref{line:kth:if2} tratam o caso em que estamos buscando, respectivamente, o primeiro e o último elemento, e esses estão no prefixo ou sufixo. Caso contrário, o~$k$-ésimo elemento está em~$d.\V{center}$ e a linha~\nref{line:kth:if3} arruma~$k$ para indicar qual elemento de~$d.\V{center}$ deve ser buscado, já que, se~$d.\V{prefix}$ é não nulo, temos que encontrar o~$(k-1)$-ésimo elemento dos que estão em~$d.\V{center}$.

Como~$d.\V{center}$ é uma deque de pares, usamos~\funcAPI{k-th}{d.\V{center}, \ceil*{\frac{k}{2}}} para encontrar o~$\ceil*{\frac{k}{2}}$-ésimo par desta deque, pois o primeiro par guarda os elementos 1 e 2, o segundo guarda os elementos 3 e 4, e assim por diante. Se~$k$ é ímpar, queremos o primeiro elemento desse par, senão o segundo. De fato, se~$k$ é ímpar, o par tem os elementos com índice~$k$ e~$k+1$, se não tem os elementos com índice~$k-1$ e~$k$.
\end{proof}

A operação consome tempo~$\Oh(\lg n)$, onde~$n$ é o tamanho da deque, pois este é o número máximo de níveis da estrutura recursiva. A Tabela~\ref{tab:deque2_persist} mostra o consumo de tempo e espaço da implementação discutida neste capítulo.

\begin{table} \centering
\begin{tabular}{|l|c|c|}
	\hline
	& Tempo/Espaço \\ \hline
	\funcAPI{Deque}{} & $\Oh(1) / \Oh(1)$ \\
	\funcAPI{PushFront}{q, x} & $\Oh(\lg n) / \Oh(\lg n)$ \\
	\funcAPI{PushBack}{q, x} & $\Oh(\lg n) / \Oh(\lg n)$ \\
	\funcAPI{Front}{q} & $\Oh(\lg n)$ \\
	\funcAPI{Back}{q} & $\Oh(\lg n)$ \\
	\funcAPI{PopFront}{q} & $\Oh(\lg n) / \Oh(\lg n)$ \\
	\funcAPI{PopBack}{q} & $\Oh(\lg n) / \Oh(\lg n)$ \\
	\funcAPI{k-th}{q, k} & $\Oh(\lg n)$ \\ \hline
\end{tabular}
	\caption{Consumo de tempo e espaço da implementação deste capítulo, onde~$n$ é o tamanho atual da deque. \label{tab:deque2_persist}}
\end{table}

\end{document}
