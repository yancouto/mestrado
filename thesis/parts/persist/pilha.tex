\documentclass[main.tex]{subfiles}

\begin{document}

\chapter{Pilhas e filas de acesso aleatório} \label{cap:pilha_persist}

Em nosso pseudocódigo, usaremos estruturas de dados como objetos. As operações aplicadas sobre uma determinada ED são funções que recebem essa ED como argumento (além de possíveis argumentos adicionais). Sempre existe uma operação que devolve uma nova instância dessa ED vazia. As operações são o meio com o qual o cliente lida com a ED.
Chamamos de efêmera uma implementação usual de uma ED, que não é necessariamente persistente.

Para manipular estruturas persistentes, é conveniente que, ao chamar uma operação de modificação, uma nova versão da ED seja devolvida, e a versão anterior continue disponível, ou seja, continue podendo ser usada tanto para operações de acesso como de modificação. A maneira como a versão anterior está armazenada pode mudar, mas a resposta para qualquer operação sobre ela deve ser a mesma. Dizemos que a versão atual da ED é a última versão criada por alguma operação de modificação.

Como aquecimento, a seguir apresentamos implementações persistentes de duas estruturas de dados simples: pilhas e filas. A descrição de pilhas persistentes, incluindo a operação~\API{k-th}, foi feita por Myers~\cite{Myers83}.

\section{Pilhas persistentes}

Pilhas são uma das estruturas de dados mais simples. Usualmente as seguintes operações estão disponíveis para se manipular uma pilha:

\begin{itemize}
	\item \funcAPI{Stack}{}
		\\ Devolve uma pilha vazia.
	\item \funcAPI{Push}{p, x}
		\\ Devolve uma cópia da pilha~$p$ com o valor~$x$ inserido no seu topo.
	\item \funcAPI{Pop}{p}
		\\ Devolve uma cópia da pilha~$p$ com o elemento do seu topo removido.
	\item \funcAPI{Size}{p}
		\\ Devolve o número de elementos na pilha~$p$.
	\item \funcAPI{Top}{p}
		\\ Devolve o elemento do topo da pilha~$p$.
\end{itemize}

Uma pilha é de \deff{acesso aleatório} se permite acesso a qualquer elemento seu, não apenas a seu topo. Para que os acessos a uma tal pilha estejam bem definidos, consideramos seus elementos na ordem em que foram inseridos. Dessa maneira, o topo da pilha é o seu último elemento e a seguinte operação está também disponível:

\begin{itemize}
	\item \funcAPI{k-th}{p, k}
		\\ Devolve o $k$-ésimo elemento da pilha~$p$.
\end{itemize}

Observe que as operações~\API{Push} e~\API{Pop} são de modificação enquanto que~\API{Top},~\API{k-th} e~\API{Size} são operações de acesso. No resto desta seção descreveremos a implementação de uma pilha de acesso aleatório totalmente persistente. Para deixar clara a notação, operações são~\API{sublinhadas} e funções auxiliares (que usamos para nos ajudar a implementar as operações) não são.

\subsection{Persistência total}

Para implementar uma pilha persistente, utilizaremos a implementação de pilhas usando lista ligada. Essa implementação consiste de vários nós, cada um com três campos:~\code{val}, com o valor armazenado neste nó,~\code{next}, com um ponteiro para o próximo nó na lista e~\code{size} com o tamanho da lista começando naquele nó. O último nó da lista ligada tem como seu campo~\code{next} um valor especial~\keyword{null}, que indica que este é o último nó da lista.

Para que as operações tenham implementações eficientes, os elementos da pilha são armazenados em uma lista ligada na ordem inversa, ou seja, o último elemento da pilha (seu topo) é armazenado no primeiro nó da lista ligada. A pilha é representada por um ponteiro para esse primeiro nó da lista ligada, ou para~\keyword{null}, se a pilha está vazia.

Para realizar a operação~\funcAPI{Push}{p, x}, basta criar um novo nó com o valor~$x$ e inseri-lo no início da lista ligada. O novo nó passa a ser o primeiro nó da lista ligada. Na implementação de uma pilha efêmera, para realizar a operação~\funcAPI{Pop}{p}, basta atualizar o ponteiro da pilha para o segundo nó da lista. O primeiro nó pode ser devolvido ou descartado.

Em princípio, não é necessário alterar os campos dos nós já criados. Em particular, poderíamos manter todos os nós já criados, sem alterá-los. Dessa forma, se guardarmos em~$p_i$ um ponteiro para o primeiro nó da lista ligada após a~$i$-ésima operação de modificação, o ponteiro~$p_i$ nos daria acesso a~$i$-ésima versão da pilha. De fato, os nós acessíveis a partir de~$p_i$ não mudarão mesmo após futuras operações. Portanto é possível realizar operações de acesso e modificação em versões anteriores da pilha, resultando em uma implementação persistente dessa.

Implementações de estruturas como esta, nas quais operações de modificação não modificam nenhum valor existente da ED, apenas criam valores novos, são chamadas de~\deff{funcionais}. Como qualquer valor acessível a partir de uma versão antiga continua o mesmo, é fácil obter uma implementação persistente para tais estruturas. Note que, diferente de implementações usuais, não é permitido apagar valores que não serão mais usados na versão atual. Dizemos que a versão atual é a última versão criada por alguma operação de modificação.

O Código~\ref{lst:pilhapersist} mostra a implementação de tal pilha persistente. Considere que~{\keyword{new}~\func{Node}{x, nx, sz}} cria um novo nó com campos~$\V{val} = x$,~$\V{next} = \V{nx}$ e~$\V{size} = \V{sz}$. Note que ignoramos a operação~\funcAPI{k-th}{p, k} por ora, mas voltaremos a discuti-la na Subseção~\ref{subsec:pilha_persist_kth}.


\begin{algorithm}
\caption{Pilha persistente.} \label{lst:pilhapersist}
\begin{algorithmic}[1]

\Function{\API{Stack}}{\null}
	\State \Return \Null
\EndFunction

\Function{\API{Size}}{$p$}
	\If{$p = \Null$}
		\State \Return $0$
	\Else
		\State \Return $p.\V{size}$
	\EndIf
\EndFunction

\Function{\API{Push}}{$p, x$}
	\State \Return \New \func{Node}{x, p, \funcAPI{Size}{p} + 1}
\EndFunction

\Function{\API{Top}}{$p$}
	\State \Return $p.\V{val}$
\EndFunction

\Function{\API{Pop}}{$p$}
	\State \Return $p.\V{next}$
\EndFunction

\end{algorithmic}
\end{algorithm}

\subsection{Exemplo}

Vamos considerar a sequência de operações no Exemplo~\ref{ex:pilha}. Na esquerda temos as operações realizadas, e na direita as novas pilhas criadas, ou o valor devolvido pela operação~\funcAPI{Top}{p}.

\begin{example}
\centering

\begin{subalgorithm}{0.4\textwidth}
\begin{algorithmic}

\State $p_0 =$ \funcAPI{Stack}{}
\State $p_1 =$ \funcAPI{Push}{p_0, 5}
\State $p_2 =$ \funcAPI{Push}{p_1, 7}
\State $p_3 =$ \funcAPI{Push}{p_2, 6}
\State $p_4 =$ \funcAPI{Pop}{p_2}
\State \funcAPI{Top}{p_3}
\State $p_5 =$ \funcAPI{Push}{p_4, 9}
\State \funcAPI{Top}{p_4}
\State $p_6 =$ \funcAPI{Push}{p_0, 5}

\end{algorithmic}
\end{subalgorithm} \vrule
\begin{subalgorithm}{0.4\textwidth}
\begin{algorithmic}

\State $p_0:$
\State $p_1:$ 5
\State $p_2:$ 5 7
\State $p_3:$ 5 7 6
\State $p_4:$ 5
\State Devolve 6
\State $p_5:$ 5 9
\State Devolve 5
\State $p_6:$ 5

\end{algorithmic}
\end{subalgorithm}
\caption{Exemplo de uso de uma pilha persistente.} \label{ex:pilha}
\end{example}

Em uma pilha efêmera, adicionamos apenas elementos no início da lista ligada, mas no caso da pilha persistente podemos adicionar nós em outros pontos, e na verdade a estrutura resultante é uma arborescência, ou seja, uma árvore enraizada onde as arestas apontam em direção à raiz, considerando que a raiz é o valor~\keyword{null} (pilha vazia). Note que, como discutido anteriormente, a partir de cada nó podemos apenas acessar os nós do caminho deste nó até a raiz, utilizando seus campos~\code{next}.

A Figura~\ref{fig:pilha_ex1} mostra a arborescência criada para a sequência de operações do Exemplo~\ref{ex:pilha}. Em cada nó é indicado o valor armazenado naquele nó, e a flecha saindo de cada nó indica seu campo~$\V{next}$. Os valores~$p_0, \ldots, p_6$ apontam para os seus nós correspondentes. Note que é possível que~${p_i = p_j}$ para~${i \neq j}$. Isso ocorre no exemplo, onde~$p_4 = p_1$. Isso não ocorre sempre que as pilhas têm os mesmos valores; perceba que~$p_1 \neq p_6$, apesar destas duas pilhas terem os mesmos elementos na mesma ordem.

Perceba também que a árvore de versões não é igual à árvore da estrutura. A Figura~\ref{fig:pilha_ex2} mostra a árvore de versões para essa sequência de operações. Nos nós estão os índices das versões~(a~$i$-ésima modificação cria a versão~$i$), e a versão~$i$ é o pai da versão~$j$ se a versão~$j$ foi criada usando a versão~$i$.

\begin{figure}
	\centering
	\begin{tikzpicture}[sibling distance=25pt, nodes={draw, minimum size=6mm}, edge from parent/.append style={<-, shorten <= 5pt, shorten >= 5pt}, level distance=40pt]
		\Tree [.\node[minimum size=6mm](root){};
			[.\node(p14){5}; [.\node(p2){7}; [.\node(p3){6}; ] ] \node(p5){9}; ]
			\node(p6){5};
		]
\draw (root.south west) edge (root.north east);
\draw[->, shorten >= 10pt] node[draw=none, left = 1cm of root]{$p_0$} edge (root);

\draw[->, shorten >= 10pt] node[draw=none, left = 1cm of p14,yshift=12pt]{$p_1$} edge (p14);
\draw[->, shorten >= 10pt] node[draw=none, left = 1cm of p14]{$p_4$} edge (p14);
\draw[->, shorten >= 10pt] node[draw=none, left = 1cm of p2]{$p_2$} edge (p2);
\draw[->, shorten >= 10pt] node[draw=none, left = 1cm of p3]{$p_3$} edge (p3);
\draw[->, shorten >= 10pt] node[draw=none, right = 1cm of p5]{$p_5$} edge (p5);
\draw[->, shorten >= 10pt] node[draw=none, right = 1cm of p6]{$p_6$} edge (p6);

	\end{tikzpicture}
	\caption{Arborescência criada pela sequência de operações do Exemplo~\ref{ex:pilha}. A raiz na figura (o nó cortado) é a representação do valor~\keyword{null}.} \label{fig:pilha_ex1}
\end{figure}

\begin{figure}
	\centering
	\begin{tikzpicture}[sibling distance=15pt]
		\Tree [.0
			[.1 [.2 3 [.4 5 ] ] ]
			6
		]
	\end{tikzpicture}
	\caption{Árvore de versões associada à sequência de operações do Exemplo~\ref{ex:pilha}.} \label{fig:pilha_ex2}
\end{figure}


\subsection{Acesso a outros elementos} \label{subsec:pilha_persist_kth}

Apresentaremos a implementação da função~\funcAPI{k-th}{p, k}. Esta operação é uma generalização de~\funcAPI{Top}{p}, já que~${\funcAPI{Top}{p} = \funcAPI{k-th}{p, p.\V{size}}}$.

Lembre-se que~$p$ é um nó de uma arborescência, e representa o último elemento da pilha. Logo~\funcAPI{k-th}{p, k} precisa determinar o~$(p.\V{size}-k)$-ésimo ancestral de~$p$, ou seja, o nó alcançado a partir de~$p$ ao caminhar por~$p.\V{size} - k$ links~$\V{next}$. Nesse caso, o~0-ésimo ancestral de um nó é o próprio nó.

Note que esse é o problema do Ancestral de Nível, discutido nos Capítulos~\ref{cap:ancestrais} e~\ref{cap:skew}. Nesses capítulos foi discutido como encontrar ancestrais de níveis usando as operações~\funcAPI{AddLeaf}{u} e~\funcAPI{LevelAncestor}{k, u}. Note que apresentamos essas operações de forma que folhas novas possam ser adicionadas de forma online, o que é necessário neste problema, já que adicionamos uma folha quando aplicamos uma operação~\API{Push} à pilha.

\providecommand{\Par}{\operatorname{Parent}}

Se assumimos que a função~${\New \func{Node}{x, \V{nx}, \V{sz}}}$ também chama a operação~\funcAPI{AddLeaf}{u} para o novo nó criado, com~$\Par(u) = \V{nx}$, então o Código~\ref{lst:pilha-kth} implementa a operação~\funcAPI{k-th}{p, k}, e está trivialmente correto.

\begin{algorithm}
\begin{algorithmic}[1]
\Function{\API{k-th}}{$p, k$}
	\State \Return $\Call{\API{LevelAncestor}}{p.\V{size} - k, p}.\V{val}$
\EndFunction
\end{algorithmic}
\caption{Implementação de~\funcAPI{k-th}{p, k} usando~Ancestral de Nível como caixa preta.} \label{lst:pilha-kth}
\end{algorithm}

A Tabela~\ref{tab:pilha_persist} mostra o consumo de tempo e espaço de uma pilha persistente, usando o Ancestral de Nível dos Capítulos~\ref{cap:ancestrais} e~\ref{cap:skew}.

\begin{table} \centering
\begin{tabular}{|l|c|c|}
	\hline
	& Representação Binária & Skew-Binary \\ \hline
	\funcAPI{Stack}{} & $\Oh(1) / \Oh(1)$ & $\Oh(1) / \Oh(1)$ \\
	\funcAPI{Push}{p, x} & $\Oh(\lg n) / \Oh(\lg n)$ & $\Oh(1) / \Oh(1)$ \\
	\funcAPI{Pop}{p} & $\Oh(1)$ & $\Oh(1)$ \\
	\funcAPI{Top}{p} & $\Oh(1)$ & $\Oh(1)$ \\
	\funcAPI{Size}{p} & $\Oh(1)$ & $\Oh(1)$ \\
	\funcAPI{k-th}{p, k} & $\Oh(\lg n)$ & $\Oh(\lg n)$ \\ \hline
\end{tabular}
	\caption{Consumo de tempo e espaço da implementação de uma pilha de acesso aleatório persistente, usando cada uma das implementações de Ancestral de Nível, onde~$n$ é o tamanho atual da pilha. \label{tab:pilha_persist}}
\end{table}

\section{Filas persistentes}

Nesta seção apresentaremos uma implementação persistente bastante natural que desenvolvemos para filas, usando uma pilha persistente. A ideia dessa implementação surgiu quando elaboramos um problema para a Seletiva USP de 2016, uma prova que seleciona alunos da USP para participar da Maratona de Programação. O problema em inglês pode ser acessado em~\href{http://codeforces.com/gym/101064/problem/G}{\texttt{codeforces.com/gym/101064/problem/G}}.

Filas são estruturas quase tão simples quanto pilhas. Elas são listas, e permitem inserções no final e remoções no início. Mais precisamente, filas de acesso aleatório permitem as seguintes operações:

\begin{itemize}
	\item \funcAPI{Queue}{}
		\\ Devolve uma fila vazia.
	\item \funcAPI{Enqueue}{q, x}
		\\ Devolve uma cópia da fila~$q$ com o valor~$x$ inserido em seu fim.
	\item \funcAPI{Dequeue}{q}
		\\ Devolve uma cópia da fila~$q$ com seu primeiro elemento removido.
	\item \funcAPI{Size}{q}
		\\ Devolve o número de elementos na fila~$q$.
	\item \funcAPI{First}{q}
		\\ Devolve o primeiro elemento da fila~$q$.
	\item \funcAPI{k-th}{q, k}
		\\ Devolve o~\mbox{$k$-ésimo} elemento da fila~$q$.
\end{itemize}

É possível, como com pilhas, implementar uma fila usando lista ligada. A lista torna-se uma árvore quando se adiciona a persistência. Seria então necessário usar a solução para o problema do~Ancestral de Nível nas operações~\funcAPI{k-th}{q, k} e~\funcAPI{Last}{q}.
Utilizaremos, entretanto, as próprias funções de pilha para implementar uma fila.

Uma fila é representada por um par em que o primeiro elemento é uma pilha e o segundo é o número de elementos que já foram removidos da fila. No pseudocódigo, usaremos que~$(p,\,r)$ cria um par ordenado contendo a pilha~$p$ e um inteiro~$r$, e todo par tem os campos~$\V{stack}$ e~$\V{rem}$ contendo cada um de seus elementos.
Assumimos que criar pares consome tempo constante.


\begin{algorithm}
\begin{algorithmic}[1]
\Function{\API{Queue}}{\null}
	\State \Return $(\Null,\,0)$
\EndFunction

\Function{\API{Enqueue}}{$q, x$}
	\State \Return $(\Call{Push}{q.\V{stack}, x},\,q.\V{rem})$
\EndFunction

\Function{\API{Dequeue}}{$q$}
	\State \Return $(q.\V{stack},\,q.\V{rem} + 1)$
\EndFunction

\Function{\API{Size}}{$q$}
	\State \Return $\funcAPI{Size}{q.\V{stack}} - q.\V{rem}$ \Comment{Função homônima da pilha}
\EndFunction

\Function{\API{First}}{$q$}
	\State \Return $\funcAPI{k-th}{q.\V{stack}, q.\V{rem} + 1}$ \Comment{Função homônima da pilha}
\EndFunction

\Function{\API{k-th}}{$q, k$}
	\State \Return $\funcAPI{k-th}{q.\V{stack}, q.\V{rem} + k}$ \Comment{Função homônima da pilha}
\EndFunction

\end{algorithmic}
\caption{Fila de acesso aleatório persistente.} \label{lst:fila}
\end{algorithm}

O Código~\ref{lst:fila} mostra a implementação das operações da fila. A correção segue diretamente da correção das funções para pilha. De fato, podendo acessar qualquer elemento de uma pilha, implementar uma fila é simples, já que usamos apenas as operações~\textsc{Push} e~\textsc{k-th}, e simplesmente ignoramos os elementos que existem antes do início da fila.

%Sempre assumimos que todas as chamadas de operações são válidas, ou seja, nunca haverá uma chamada para~\funcAPI{k-th}{q, k} com~$k$ maior que o tamanho de~$q$, de modo que os elementos antes do início da fila nunca serão acessados indevidamente.

Desde que a implementação da pilha seja persistente, a implementação da fila é persistente pois nunca modifica nenhum par, apenas cria novos objetos. A Tabela~\ref{tab:fila_persist} mostra o consumo de tempo e espaço da implementação discutida neste capítulo.

\begin{table} \centering
\begin{tabular}{|l|c|c|}
	\hline
	& Representação Binária & Skew-Binary \\ \hline
	\funcAPI{Queue}{} & $\Oh(1) / \Oh(1)$ & $\Oh(1) / \Oh(1)$ \\
	\funcAPI{Enqueue}{q, x} & $\Oh(\lg n) / \Oh(\lg n)$ & $\Oh(1) / \Oh(1)$ \\
	\funcAPI{Dequeue}{q} & $\Oh(1)$ & $\Oh(1)$ \\
	\funcAPI{Size}{q} & $\Oh(1)$ & $\Oh(1)$ \\
	\funcAPI{First}{q} & $\Oh(\lg n)$ & $\Oh(\lg n)$ \\
	\funcAPI{k-th}{q, k} & $\Oh(\lg n)$ & $\Oh(\lg n)$ \\ \hline
\end{tabular}
	\caption{Consumo de tempo e espaço da implementação de uma fila de acesso aleatório persistente, usando cada uma das implementações de pilha persistente, onde~$n$ é o tamanho atual da fila. \label{tab:fila_persist}}
\end{table}

\end{document}
