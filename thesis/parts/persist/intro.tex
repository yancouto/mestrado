\documentclass[../../main.tex]{subfiles}

\begin{document}

\setcounter{secnumdepth}{0}

\chapter*{Introdução}
\chaptermark{Introdução}
\addcontentsline{toc}{chapter}{Introdução}

Uma estrutura de dados (ED) é uma forma de organizar dados em programas de computador. Estruturas de dados permitem operações de acesso e de modificação; operações de acesso apenas consultam um ou mais campos de uma ED, enquanto operações de modificação podem alterar os campos da estrutura.

Em geral, operações de acesso só podem ser feitas na configuração atual da ED, ou seja, ao realizar uma operação de modificação, perde-se informação sobre o ``passado'' da estrutura. Dizemos que, ao realizar uma operação de modificação, criamos uma nova versão da ED. Estruturas de dados \deff{persistentes}~\cite{DriscollSST1989} permitem realizar operações em versões criadas anteriormente. Dizemos que uma ED é \deff{parcialmente persistente} se permite apenas operações de acesso a versões anteriores e modificação apenas na versão mais nova, e \deff{totalmente persistente}, ou apenas~\deff{persistente}, se também permite operações de modificação em todas as versões.

Considerando um digrafo das versões onde se a versão $j$ foi criada a partir da versão $i$ então existe um arco de $i$ para $j$, temos que para estruturas parcialmente persistentes esse digrafo é um caminho, e para estruturas totalmente persistentes é uma árvore enraizada, em que as arestas vão para longe da raiz.

O estudo de estruturas persistentes segue dois caminhos: técnicas gerais, para tornar \emph{qualquer} estrutura de dados persistente, ou técnicas para tornar alguma ED específica (como uma pilha ou fila) persistente, deixando-a tão eficiente e simples quanto possível.

\setcounter{secnumdepth}{1}

\end{document}
