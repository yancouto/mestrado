\documentclass[main.tex]{subfiles}

\begin{document}

\chapter{Deque com LA e LCA} \label{cap:deque1_persist}

Uma~\deff{fila com duas pontas} é uma estrutura de dados que generaliza pilhas e filas. Essa estrutura é uma lista em que é possível adicionar e remover elementos de qualquer uma das suas extremidades. Ela é usualmente chamada de~\deff{deque}, uma abreviatura de~\deff{dequeue}.
Chama-se uma das extremidades da deque de início e outra de fim, embora ambas tenham o mesmo papel.

Uma deque de acesso aleatório admite as seguintes operações:

\begin{itemize}
	\item \funcAPI{Deque}{}
	      \\ Devolve uma deque vazia.
	\item \funcAPI{Front}{d}
	      \\ Devolve o primeiro elemento de~$d$.
	\item \funcAPI{Back}{d}
	      \\ Devolve o último elemento de~$d$.
	\item \funcAPI{PushFront}{d, x}
	      \\ Devolve uma cópia da deque~$d$ com o valor~$x$ inserido no seu início.
	\item \funcAPI{PushBack}{d, x}
	      \\ Devolve uma cópia da deque~$d$ com o valor~$x$ inserido no seu fim.
	\item \funcAPI{PopFront}{d}
	      \\ Devolve uma cópia da deque~$d$ sem o primeiro elemento.
	\item \funcAPI{PopBack}{d}
	      \\ Devolve uma cópia da deque~$d$ sem o último elemento.
	\item \funcAPI{k-th}{d, k}
	      \\ Devolve o $k$-ésimo elemento da deque~$d$.
\end{itemize}

Note que, usando apenas \funcAPI{PushBack}{d, x}, \funcAPI{Back}{d} e \funcAPI{PopBack}{d}, podemos simular uma pilha. Analogamente, usando apenas \funcAPI{PushBack}{d, x}, \funcAPI{Front}{d} e \funcAPI{PopFront}{d}, podemos simular uma fila simples.

Na literatura existem outras implementações persistentes de deques, que serão apresentadas nos dois próximos capítulos. Neste capítulo, apresentaremos uma implementação persistente que desenvolvemos de deques. Ela se baseia nas implementações de pilhas e filas persistentes discutidas no Capítulo~\ref{cap:pilha_persist}, e utiliza Ancestral de Nível e Ancestral Comum Mais Profundo, discutidos nos Capítulos~\ref{cap:ancestrais} e~\ref{cap:skew}.

Embora utilize os algoritmos para Ancestral de Nível e Ancestral Comum Mais Profundo, acreditamos que esta implementação seja mais simples que as apresentadas nos capítulos seguintes.

\section{Representação e visão geral}

Na implementação de pilhas no Capítulo~\ref{cap:pilha_persist}, a lista ligada, quando em um contexto persistente, se torna uma arborescência. É possível adicionar novas folhas, pois estas não mudam os ponteiros dos outros nós; e então é possível adicionar elementos ao final da pilha.

Na implementação de filas persistentes, descrita no mesmo capítulo, também foi discutido como simular a remoção de elementos do início da estrutura, guardando, junto ao nó indicado pela versão da fila, o número de elementos já removidos naquela versão. Dessa forma, os elementos de uma determinada versão da fila são um sufixo do caminho da raiz até o nó indicado pela versão. Essa organização, porém, não permite adicionar elementos no início da fila.

\begin{figure}
\centering
\begin{tikzpicture}[sibling distance=40pt, nodes={draw, minimum size=6mm}, edge from parent/.append style={<-, shorten <= 5pt, shorten >= 5pt}, level distance=40pt, level 3/.style={sibling distance=80pt}, level 1/.style={sibling distance=80pt}]

\Tree [.\node[minimum size=6mm](root){};
    [.\node(v3){3}; [.\node(v2){2}; \node(v1){1}; [.\node(v9){9}; ] ] \node(v4){4}; ]
    \node(v6){6};
]

\draw (root.south west) edge (root.north east);
\draw[->, shorten >= 10pt] node[draw=none, left = 1cm of root,yshift=12pt]{$\V{first}_0$} edge (root);
\draw[->, shorten >= 10pt] node[draw=none, right = 1cm of root,yshift=12pt]{$\V{last}_0$} edge (root);

\draw[->, shorten >= 10pt] node[draw=none, left = 1cm of v3,yshift=12pt]{$\V{first}_1$} edge (v3);
\draw[->, shorten >= 10pt] node[draw=none, right = 1cm of v3,yshift=12pt]{$\V{last}_1$} edge (v3);

\draw[->, shorten >= 10pt] node[draw=none, left = 1cm of v3]{$\V{first}_2$} edge (v3);
\draw[->, shorten >= 10pt] node[draw=none, right = 1cm of v4,yshift=12pt]{$\V{last}_2$} edge (v4);

\draw[->, shorten >= 10pt] node[draw=none, left = 1cm of v2,yshift=12pt]{$\V{first}_3$} edge (v2);
\draw[->, shorten >= 10pt] node[draw=none, right = 1cm of v4]{$\V{last}_3$} edge (v4);

\draw[->, shorten >= 10pt] node[draw=none, left = 1cm of v1]{$\V{first}_4$} edge (v1);
\draw[->, shorten >= 10pt] node[draw=none, right = 1cm of v4,yshift=-12pt]{$\V{last}_4$} edge (v4);

\draw[->, shorten >= 10pt] node[draw=none, left = 1cm of v2]{$\V{first}_5$} edge (v2);
\draw[->, shorten >= 10pt] node[draw=none, right = 1cm of v3]{$\V{last}_5$} edge (v3);

\draw[->, shorten >= 10pt] node[draw=none, left = 1cm of v2,yshift=-12pt]{$\V{first}_6$} edge (v2);
\draw[->, shorten >= 10pt] node[draw=none, right = 1cm of v2,yshift=12pt]{$\V{last}_6$} edge (v2);

\draw[->, shorten >= 10pt] node[draw=none, left = 1cm of v9]{$\V{first}_7$} edge (v9);
\draw[->, shorten >= 10pt] node[draw=none, right = 1cm of v2]{$\V{last}_7$} edge (v2);

\draw[->, shorten >= 10pt] node[draw=none, left = 1cm of root]{$\V{first}_8$} edge (root);
\draw[->, shorten >= 10pt] node[draw=none, right = 1cm of root]{$\V{last}_8$} edge (root);

\draw[->, shorten >= 10pt] node[draw=none, left = 1cm of v6]{$\V{first}_9$} edge (v6);
\draw[->, shorten >= 10pt] node[draw=none, right = 1cm of v6]{$\V{last}_9$} edge (v6);

\end{tikzpicture}
\caption{Arborescência criada pela sequência de operações do Exemplo~\ref{ex:deque}. A raiz na figura (o nó cortado) é a representação do valor~\keyword{null}.} \label{fig:deque_ex1}
\end{figure}

\begin{example}[t]
\centering

\begin{subalgorithm}{0.55\textwidth}
\begin{algorithmic}

\State $(\V{first}_0, \V{last}_0) = \Call{Deque}{\null}$
\State $(\V{first}_1, \V{last}_1) = \Call{PushBack}{(\V{first}_0,\V{last}_0), 3}$
\State $(\V{first}_2, \V{last}_2) = \Call{PushBack}{(\V{first}_1,\V{last}_1), 4}$
\State $(\V{first}_3, \V{last}_3) = \Call{PushFront}{(\V{first}_2,\V{last}_2), 2}$
\State $(\V{first}_4, \V{last}_4) = \Call{PushFront}{(\V{first}_3,\V{last}_3), 1}$
\State $(\V{first}_5, \V{last}_5) = \Call{PopBack}{(\V{first}_3,\V{last}_3)}$
\State $(\V{first}_6, \V{last}_6) = \Call{PopBack}{(\V{first}_5,\V{last}_5)}$
\State $(\V{first}_7, \V{last}_7) = \Call{PushFront}{(\V{first}_6,\V{last}_6), 9}$
\State $(\V{first}_8, \V{last}_8) = \Call{PopFront}{(\V{first}_6,\V{last}_6)}$
\State $(\V{first}_9, \V{last}_9) = \Call{PushFront}{(\V{first}_8,\V{last}_8), 6}$

\end{algorithmic}
\end{subalgorithm} \vrule
\begin{subalgorithm}{0.4\textwidth}
\begin{algorithmic}

\State $\Rightarrow$
\State $\Rightarrow$ 3
\State $\Rightarrow$ 3 4
\State $\Rightarrow$ 2 3 4
\State $\Rightarrow$ 1 2 3 4
\State $\Rightarrow$ 2 3
\State $\Rightarrow$ 2
\State $\Rightarrow$ 9 2
\State $\Rightarrow$
\State $\Rightarrow$ 6

\end{algorithmic}
\end{subalgorithm}
\caption{Exemplo de uso de uma deque persistente.} \label{ex:deque}
\end{example}

Para a implementação de deques, ainda usaremos uma arborescência, mas cada versão da deque indicará dois nós:~$\V{first}$ e~$\V{last}$. Os elementos dessa versão da deque serão os elementos no caminho entre~$\V{first}$ para~$\V{last}$ na árvore subjacente à arborescência (ou seja, na versão não dirigida da arborescência). Na Figura~\ref{fig:deque_ex1}, a deque apontada pelo par~$(\V{first}_3, \V{last}_3)$, por exemplo, contém os elementos~2,~3 e~4.

Para adicionar um elemento no início de uma determinada versão da deque, criamos uma nova folha conectada ao~$\V{first}$ dessa versão e indicamos essa nova folha como o~$\V{first}$ da nova versão criada e o seu~$\V{last}$ é o mesmo da versão que foi modificada; para adicionar um elemento no final de uma determinada versão da deque, criamos uma nova folha conectada a~$\V{last}$ e definimos o~$\V{first}$ e~$\V{last}$ da nova versão analogamente. Criar folhas em uma árvore não muda os caminhos entres vértices já existentes, então as versões anteriores da deque continuam válidas, e o novo par~$(\V{first}, \V{last})$ representa a deque alterada com um elemento adicionado em seu início ou final. Isso funciona se a versão da deque em questão tem pelo menos um elemento antes da adição. Na Figura~\ref{fig:deque_ex1}, ao inserir~1 no início da deque representada por~$(\V{first}_3, \V{last_3})$, obtemos a deque~$(\V{first}_4, \V{last}_4)$.

Para remover um elemento do início de uma versão da deque indicada pelo par~$(\V{first}, \V{last})$, tomamos como~$\V{first}$ da nova versão o segundo elemento no caminho entre~$\V{first}$ e~$\V{last}$; para remover um elemento do final dessa versão, tomamos como~$\V{last}$ o vizinho de~$\V{last}$ no caminho entre~$\V{first}$ e~$\V{last}$. Isso functiona se a versão da deque em questão tem pelo menos dois elementos antes da remoção. Como fazer essas operações será discutido nas próximas seções. Na Figura~\ref{fig:deque_ex1}, ao remover o último elemento da deque~$(\V{first}_5, \V{last}_5)$ obtemos a deque~$(\V{first}_6, \V{last}_6)$.

É necessária uma representação especial para deques vazias, por exemplo, o par~$(\keyword{null}, \keyword{null})$. Assim, ao remover um elemento de uma deque com um único elemento (ou seja, quando~$\V{first}=\V{last}$), devolvemos o par~$(\keyword{null}, \keyword{null})$; e para adicionar algum elemento à deque vazia basta criarmos um nó~$u$ com o valor desejado e devolvermos o par~$(u, u)$. Dessa forma nossa estrutura é na verdade uma coleção de arborescências, ou podemos considerar que é uma arborescência em~\keyword{null} representa a raiz, assim como fizemos para pilhas no Capítulo~\ref{cap:pilha_persist}. Na Figura~\ref{fig:deque_ex1}, ao remover o único elemento da deque~$(\V{first}_6, \V{last}_6)$, obtemos a deque vazia~$(\V{first}_8, \V{last}_8)$, onde o nó cortado representa~\keyword{null}, e ao adicionar~6 a essa deque obtemos~$(\V{first}_9, \V{last}_9)$.

\section{Acesso e inserção}

Em cada nó da arborescência guardaremos: o pai do nó no campo~$\V{parent}$; sua profundidade no campo~$\V{depth}$; e seu valor associado, que é um dos valores armazenados na deque, no campo~$\V{val}$. O código~$\New \Call{Node}{x, p, d}$ cria um novo nó com valores~$x$,~$p$ e~$d$ nos campos~$\V{val}$,~$\V{parent}$ e~$\V{depth}$, respectivamente, e também realiza o pré-processamento necessário para o algoritmo de Ancestral de Nível funcionar, ou seja, chama a função~\API{AddLeaf} para o novo vértice, como discutido nos Capítulos~\ref{cap:ancestrais} e~\ref{cap:skew}.

Uma deque é representada por um par ordenado de nós. Usaremos~$(a, b)$ para criar uma deque associada aos nós~$a$ e~$b$, e estes nós podem ser acessados pelos campos~$\V{first}$ e~$\V{last}$.

\begin{algorithm}
\caption{Operações de acesso e inserção.} \label{lst:deque1_ops1}
\begin{algorithmic}[1]

\Function{\API{Deque}}{\null}
    \State \Return $(\Null, \Null)$
\EndFunction

\Function{\API{Front}}{$d$}
    \State \Return $d.\V{first}.\V{val}$
\EndFunction

\Function{\API{Back}}{$d$}
    \State \Return $d.\V{last}.\V{val}$
\EndFunction

\Function{Swap}{$d$}
    \State \Return $(d.\V{last}, d.\V{first})$
\EndFunction

\Function{\API{PushFront}}{$d, x$}
    \If{$d.\V{first} = \Null$}
        \State $u = \New \Call{Node}{x, \Null, 1}$
        \State \Return $(u, u)$
    \Else
        \State \Return $(\New \Call{Node}{x, d.\V{first}, d.\V{first}.\V{depth} + 1}, d.\V{last})$
    \EndIf
\EndFunction

\Function{\API{PushBack}}{$d, x$}
    \State \Return $\Call{Swap}{\Call{PushFront}{\Call{Swap}{d}, x}}$
\EndFunction

\end{algorithmic}
\end{algorithm}

O Código~\ref{lst:deque1_ops1} mostra a implementação das operações mais simples. A função~$\Call{Swap}{d}$ devolve o inverso da deque~$d$ (basta inverter~$\V{first}$ e~$\V{last}$ no par associado a~$d$), e é usada para diminuir a necessidade de duplicar código entre as operações~\API{\textsc{PushFront}} e~\API{\textsc{PushBack}}, assim como~\API{\textsc{PopFront}} e~\API{\textsc{PopBack}}.

As implementações dadas são funcionais e mantêm a propriedade de que os elementos de uma deque são os valores nos nós no caminho de seu campo~$\V{first}$ para~$\V{last}$.

\section{Remoção}

Discutiremos como determinar o segundo elemento no caminho de~$\V{first}$ para~$\V{last}$. Encontrar o penúltimo é encontrar o segundo no caminho de~$\V{last}$ para~$\V{first}$, assim usaremos~\textsc{Swap} para não ter que tratar esse caso separadamente. Para uma árvore enraízada, dizemos que o~\deff{ancestral comum mais profundo} de~$\V{first}$ e~$\V{last}$ é o ancestral comum de ambos com maior profundidade. Seja~$\V{mid}$ este ancestral. Então o caminho de~$\V{first}$ para~$\V{last}$ é o caminho de~$\V{first}$ para~$\V{mid}$ concatenado com o caminho de~$\V{mid}$ para~$\V{last}$.

O caminho de~$\V{first}$ para~$\V{mid}$ é~${(\V{first}, \V{first}.\V{parent}, \ldots, \V{mid})}$; assim, se~$\V{first} \neq \V{mid}$, temos que~$\V{first}.\V{parent}$ é o segundo elemento do caminho. Se~$\V{first} = \V{mid}$ então o caminho de~$\V{first}$ para~$\V{last}$ é~${(\V{first}, \ldots, \V{last}.\V{parent}, \V{last})}$. Este caminho tem comprimento~${\V{last}.\V{depth} - \V{first}.\V{depth}}$, e então o segundo nó deste caminho é o~${(\V{last}.\V{depth} - \V{first}.\V{depth} - 1)}$-ésimo ancestral de~$\V{last}$ na arborescência, considerando que o~\mbox{0-ésimo} ancestral de um nó é ele mesmo, o primeiro é seu pai, e assim por diante.

\begin{algorithm}
\caption{Operações de remoção.} \label{lst:deque1_ops2}
\begin{algorithmic}[1]

\Function{\API{PopFront}}{$d$}
	\If{$d.\V{first} = d.\V{last}$}
		\State \Return \funcAPI{Deque}{\null}
	\ElsIf{$\funcAPI{LCA}{d.\V{first}, d.\V{last}} = d.\V{first}$}
		\State \Return $(\funcAPI{LevelAncestor}{d.\V{last}.\V{depth} - d.\V{first}.\V{depth} - 1, d.\V{last}}, d.\V{last})$
	\Else
		\State \Return $(d.\V{first}.\V{parent}, d.\V{last})$
	\EndIf
\EndFunction

\Function{\API{PopBack}}{$d$}
    \State \Return $\Call{Swap}{\Call{PopFront}{\Call{Swap}{d}}}$
\EndFunction

\end{algorithmic}
\end{algorithm}

O Código~\ref{lst:deque1_ops2} mostra a implementação das operações~\API{\textsc{PopFront}} e~\API{\textsc{PopBack}}. Usamos o predicado~${\funcAPI{LCA}{d.\V{first}, d.\V{last}} = d.\V{first}}$ para determinar se~$d.\V{first}$ é o ancestral comum mais profundo de~$d.\V{first}$ e~$d.\V{last}$. Esta condição pode também ser verificada usando apenas chamadas para~\mbox{\API{\textsc{LevelAncestor}}}. Se~$\V{first}$ é o ancestral comum mais profundo de~$\V{first}$ e~$\V{last}$, então o caminho de~$\V{first}$ para~$\V{last}$ é~${(\V{first}, \ldots, \V{last}.\V{parent}, \V{last})}$, mas neste caso~$d.\V{first}$ é o~${(\V{last}.\V{depth} - \V{first}.\V{depth})}$-ésimo ancestral de~$\V{last}$. Portanto $$ \funcAPI{LCA}{d.\V{first}, d.\V{last}} = d.\V{first} $$ se e somente se $$ d.\V{first}.\V{depth} < d.\V{last}.\V{depth} \AND \funcAPI{LevelAncestor}{d.\V{last}.\V{depth} - d.\V{first}.\V{depth}, d.\V{last}} = d.\V{first}. $$

\section{Acesso a outros elementos}

Para implementar a operação~\funcAPI{k-th}{d, k}, é necessário comparar~$k-1$ com o tamanho do caminho de~$\V{first}$ até~$\V{mid}$, o ancestral comum mais profundo de~$\V{first}$ e~$\V{last}$. Esse caminho tem tamanho~${\ell_1 \coloneqq \V{first}.\V{depth} - \V{mid}.\V{depth}}$. Se~$k-1$ for menor ou igual a~$\ell_1$, então o~$k$-ésimo elemento da deque é o~$(k-1)$-ésimo ancestral de~$\V{first}$; senão, devemos determinar o~$(k-\ell_1)$-ésimo elemento da segunda parte do caminho. Seja~${\ell_2 \coloneqq \V{last}.\V{depth} - \V{mid}.\V{depth}}$ o tamanho do caminho de~$\V{mid}$ até~$\V{last}$. Então o~$(k-\ell_1)$-ésimo elemento da segunda parte do caminho é o~$(\ell_2 - (k-1-\ell_1)) = (\ell_2 + \ell_1 + 1 - k)$-ésimo ancestral de~$\V{last}$.

\begin{algorithm}
\caption{Operação~\textsc{k-th}.} \label{lst:deque1_ops3}
\begin{algorithmic}[1]

\Function{\API{k-th}}{$d, k$}
	\State $\V{mid} = \funcAPI{LCA}{d.\V{first}, d.\V{last}}$
	\State $\ell_1 = d.\V{first}.\V{depth} - d.\V{mid}.\V{depth}$
	\State $\ell_2 = d.\V{last}.\V{depth} - d.\V{mid}.\V{depth}$
	\If{$k - 1 \leq \ell_1$}
		\State \Return $\funcAPI{LevelAncestor}{k - 1, d.\V{first}}$
	\Else
		\State \Return $\funcAPI{LevelAncestor}{\ell_1 + \ell_2 + 1 - k, d.\V{last}}$
	\EndIf
\EndFunction

\end{algorithmic}
\end{algorithm}

O Código~\ref{lst:deque1_ops3} faz exatamente o que foi discutido, e portanto devolve corretamente o $k$-ésimo elemento da versão~$d$ da deque, se~$k$ é válido, ou seja, está entre~1 e o número de elementos da deque~$d$. A Tabela~\ref{tab:deque1_persist} mostra o consumo de tempo e espaço da implementação discutida nesse capítulo, usando as soluções de Ancestral de Nível dos Capítulos~\ref{cap:ancestrais} e~\ref{cap:skew}.

\begin{table} \centering
\begin{tabular}{|l|c|c|}
	\hline
	& Representação binária & Skew-binary \\ \hline
	\funcAPI{Deque}{} & $\Oh(1) / \Oh(1)$ & $\Oh(1) / \Oh(1)$ \\
	\funcAPI{PushFront}{q, x} & $\Oh(\lg n) / \Oh(\lg n)$ & $\Oh(1) / \Oh(1)$ \\
	\funcAPI{PushBack}{q, x} & $\Oh(\lg n) / \Oh(\lg n)$ & $\Oh(1) / \Oh(1)$ \\
	\funcAPI{Front}{q} & $\Oh(1)$ & $\Oh(1)$ \\
	\funcAPI{Back}{q} & $\Oh(1)$ & $\Oh(1)$ \\
	\funcAPI{PopFront}{q} & $\Oh(\lg n) / \Oh(1)$ & $\Oh(\lg n) / \Oh(1)$ \\
	\funcAPI{PopBack}{q} & $\Oh(\lg n) / \Oh(1)$ & $\Oh(\lg n) / \Oh(1)$ \\
	\funcAPI{k-th}{q, k} & $\Oh(\lg n)$ & $\Oh(\lg n)$ \\ \hline
\end{tabular}
	\caption{Consumo de tempo e espaço da implementação de uma deque de acesso aleatório persistente, usando cada uma das duas implementações de Ancestral de Nível vistas na Parte~\ref{part:prelim}, onde~$n$ é o tamanho da deque. \label{tab:deque1_persist}}
\end{table}

\end{document}
