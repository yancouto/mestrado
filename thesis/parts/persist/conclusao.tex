\documentclass[main.tex]{subfiles}

\begin{document}

\setcounter{secnumdepth}{0}

\chapter*{Conclusão}
\chaptermark{Conclusão}
\addcontentsline{toc}{chapter}{Conclusão}

A dissertação apresenta persistência em estruturas de dados, mostrando vários resultados dessa área, com atenção especial para a implementação destas estruturas de forma prática.

As implementações de deque apresentadas mostram como reduzir a complexidade pode complicar bastante a teoria e implementação de uma estrutura. Como indicado na Tabela~\ref{tab:deque_persist_comp}, para reduzir a complexidade de tempo de duas operações da deque, de logarítmico pra constante, foi necessário apresentar toda a teoria e código complicados do Capítulo~\ref{cap:deque3_persist}; e, na prática, com limites razoáveis para os computadores dos dias de hoje, esta solução fica mais lenta.

Apresentamos algumas técnicas gerais para tornar algumas classes de estruturas de dados em persistentes, e aplicamos algumas destas a estruturas conhecidas: apresentamos uma deque persistente que usar implementação funcional e uma árvore rubro-negra parcialmente persistente que usa node copying.

Por último, apresentamos uma aplicação de estruturas persistentes, a solução do problema de localização de ponto. É possível generalizar este tipo de solução, na qual fazemos uma linha de varredura com alguma estrutura de dados de forma offline, já que é possível utilizar uma versão persistente da estrutura para respondermos as consultas de forma online.

\setcounter{secnumdepth}{1}

\end{document}
