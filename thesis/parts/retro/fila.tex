\documentclass[main.tex]{subfiles}

\begin{document}

\chapter{Fila retroativa}

Diferente de persistência, a estrutura mais simples de tornar retroativa é a fila. A fila que usaremos é definida como a do~Capítulo~\ref{cap:pilha_persist}. Apresentaremos neste capítulo a implementação de uma fila totalmente retroativa.

No pseudocódigo de estruturas retroativas, usaremos estas EDs como objetos que possuem funções que podem ser chamadas sobre elas. No caso da fila, por exemplo, podemos ter um objeto~$q$ e usar a função~$q.\text{Query}(10, ``\textsc{K-th}(3)")$ para acessar o terceiro elemento da fila no tempo~10. O Exemplo~\ref{ex:fila_retro} mostra um exemplo de uso de uma fila retroativa.

O motivo da fila ser a estrutura mais simples de tornar retroativa é que as operações de~\textsc{Enqueue} e~\textsc{Dequeue} são quase ``independentes''. Como~\textsc{Enqueue}s são feitos no final da estrutura e~\textsc{Dequeue}s são feitos no começo, a ordem em que~\textsc{Dequeue}s são feitos não influencia a estrutura. Note que, no Exemplo~\ref{ex:fila_retro}, se a inserção de~\textsc{Dequeue} fosse depois do segundo~\textsc{Enqueue}, o efeito na fila final seria o mesmo.

Mais formalmente, se as inserções são~$E = (e_1, \ldots, e_e)$, então a estrutura no tempo~$t$ é a lista~$(e_{d_t}, \ldots, e_{e_t})$, onde~$d_t$ é o número de remoções realizadas até o tempo~$t$ e~$e_t$ é o número de inserções realizadas até o tempo~$t$.

\begin{table}
\centering

\begin{subalgorithm}{0.34\textwidth}
\begin{algorithmic}

	\State $q =$ \func{Queue}{}
	\State $q.\text{Insert}(10, ``\Call{Enqueue}{1}")$
	\State $q.\text{Insert}(3, ``\Call{Enqueue}{2}")$
	\State $q.\text{Insert}(5, ``\Call{Dequeue}{\null}")$
	\State $q.\text{Delete}(5)$

\end{algorithmic}
\end{subalgorithm} \vrule
\begin{subalgorithm}{0.14\textwidth}
\begin{algorithmic}

	\State $q:$
	\State $q:$ 1
	\State $q:$ 2 1
	\State $q:$ 1
	\State $q:$ 2 1

\end{algorithmic}
\end{subalgorithm} \vrule
\begin{subalgorithm}{0.5\textwidth}
\begin{algorithmic}

	\State $\emptyset$
	\State $\Call{Enqueue}{1}$
	\State $\Call{Enqueue}{2} \rightarrow \Call{Enqueue}{1}$
	\State $\Call{Enqueue}{2} \rightarrow \Call{Dequeue}{\null} \rightarrow \Call{Enqueue}{1}$
	\State $\Call{Enqueue}{2} \rightarrow \Call{Enqueue}{1}$

\end{algorithmic}
\end{subalgorithm}
\caption{Exemplo de uso de uma fila retroativa. Na esquerda, as operações realizadas, no centro o estado atual da fila, e na direita a sequência de operações, ordenada por tempo.} \label{ex:fila_retro}
\end{table}

\section{Representação}

Para implementar a fila retroativa, usaremos uma estrutura que armazena um conjunto de chaves e valores, onde as chaves vem de um conjunto totalmente ordenado, e permite as seguintes operações:

\begin{itemize}
	\item $\Call{Insert}{k, v}$ --- Insere a chave~$k$ associada ao valor~$v$.
	\item $\Call{Remove}{k}$ --- Remove a chave~$k$ e seu valor associado.
	\item $\Call{K-th}{k}$ --- Devolve o valor associado à~$k$-ésima menor chave. % esquisito k ser chave e indice
	\item $\Call{Count}{k}$ --- Conta o número de chaves~$k'$ tal que~$k' \leq k$.
\end{itemize}

Essas operações podem ser implementadas usando uma ABB balanceada tal que cada operação consuma tempo logarítmico no número de elementos atualmente no conjunto, e espaço linear.

\newcommand{\deqs}{\V{deq}}
\newcommand{\enqs}{\V{enq}}

Uma fila retroativa terá dois campos,~$\enqs$ e~$\deqs$, ambos armazenam estruturas como a ED auxiliar descrita acima. O campo~$\enqs$ armazena as operações de~\textsc{Enqueue}, as chaves são o tempo e o valor é o valor inserido na fila. O campo~$\deqs$ armazena as operações de~\textsc{Dequeue}, as chaves são o tempo, e não é necessário armazenar nada no valor, logo usaremos~$\Null$ como o valor.

\end{document}
