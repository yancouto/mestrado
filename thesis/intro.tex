\documentclass[main.tex]{subfiles}

\begin{document}

\setcounter{secnumdepth}{0}

\chapter*{Introdução}
\chaptermark{Introdução}
\addcontentsline{toc}{chapter}{Introdução}

Uma estrutura de dados (ED) é uma forma de organizar dados em programas de computador. Estruturas de dados permitem operações de acesso e de modificação; operações de acesso apenas consultam um ou mais campos de uma ED, enquanto operações de modificação podem alterar os campos da estrutura.

Em geral, operações só podem ser feitas na configuração atual da ED, ou seja, ao realizar uma operação de modificação, perde-se informação sobre o ``passado'' da estrutura. Dizemos que, ao realizar uma operação de modificação, criamos uma nova versão da ED. Estruturas de dados \deff{persistentes}~\cite{DriscollSST1989} permitem realizar operações em versões criadas anteriormente. Dizemos que uma ED é \deff{parcialmente persistente} se permite apenas operações de acesso a versões anteriores e modificação apenas na versão mais nova, e \deff{totalmente persistente}, ou apenas~\deff{persistente}, se também permite operações de modificação em todas as versões.

Considerando um digrafo das versões onde, se a versão $j$ foi criada a partir da versão $i$, então existe um arco de $i$ para $j$, temos que para estruturas parcialmente persistentes esse digrafo é um caminho e, para estruturas totalmente persistentes, é uma árvore enraizada em que as arestas vão para longe da raiz. Este digrafo é chamado de árvore de versões.

O estudo de estruturas persistentes segue dois caminhos: técnicas gerais, para tornar \emph{qualquer} estrutura de dados persistente, ou técnicas para tornar alguma ED específica (como uma pilha ou fila) persistente, deixando-a tão eficiente e simples quanto possível.

Persistência foi formalmente introduzida por Driscoll, Sleator e Tarjan~\cite{DriscollSST1989}, porém já era estudada anteriormente, principalmente para a implementação de estruturas de dados em linguagens funcionais, como pilhas~\cite{Myers83}, filas~\cite{HoodMelville} e árvores de busca binária (ABBs)~\cite{Myers82}.

A Parte~\ref{part:prelim} desta dissertação detalha as soluções para os problemas de ancestral comum mais profundo e ancestral de nível, usadas como caixa preta nos Capítulos~\ref{cap:pilha_persist} e~\ref{cap:deque1_persist}. Um leitor com algum conhecimento prévio destes tópicos pode optar por iniciar a leitura desta dissertação a partir da Parte~\ref{part:persist}, consultando a Parte~\ref{part:prelim} se necessário.

A Parte~\ref{part:persist} detalha a teoria de estruturas persistentes, e se divide da seguinte forma:
\begin{itemize}
\item Os Capítulos~\ref{cap:pilha_persist} a~\ref{cap:deque3_persist} apresentam técnicas para tornar persistentes estruturas específicas, como pilhas, filas e deques.
\item O Capítulo~\ref{cap:geral_persist} apresenta técnicas gerais para tornar persistentes certas classes de estruturas de dados. O Capítulo~\ref{cap:rubronegra_persist} aplica uma destas técnicas à árvore rubro-negra.
\item O Capítulo~\ref{cap:pl_persist} apresenta uma aplicação da estrutura apresentada no Capítulo~\ref{cap:rubronegra_persist} para resolver um problema de geometria computacional conhecido como o problema de localização de ponto.
\end{itemize}

Todas as estruturas apresentadas foram implementadas em C\texttt{++}, e as implementações podem ser acessadas pelo repositório~\mbox{\href{https://github.com/yancouto/mestrado/}{\texttt{yancouto/mestrado}}} no GitHub. A documentação dessas implementações está disponível em~\mbox{\href{https://yancouto.github.io/mestrado/}{\texttt{yancouto.github.io/mestrado}}}.

\setcounter{secnumdepth}{2}

\end{document}



