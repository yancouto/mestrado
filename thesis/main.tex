\documentclass[11pt,oneside,a4paper, openany]{book}

% Meus .sty
\usepackage{formatting}
\usepackage{code}
\usepackage{theorem-vars}
\usepackage{drawing}

\title{Estruturas de dados persistentes}
\author{Yan Couto}
\date{\today}
\floatstyle{plain}
\newfloat{example}{tbhp}{loe}[chapter]
\floatname{example}{Exemplo}

\begin{document}

\frontmatter

\thispagestyle{empty}
\begin{center}
	\vspace*{2.3cm}
	\textbf{\huge{Estruturas de dados persistentes}}\\

	\vspace*{1cm}
	\Large{Yan Soares Couto}

	\vskip 1.8cm
	Orientadora: Cristina G. Fernandes\\

	\vspace{\fill}
	\normalsize{São Paulo, 2017}
\end{center}

\setcounter{tocdepth}{1}

\pagenumbering{gobble}
\begingroup
\let\cleardoublepage\clearpage
\tableofcontents
\endgroup

\mainmatter
\pagenumbering{arabic}

\part{Preliminares}
\subfile{parts/prelim/ancestrais}

\subfile{parts/prelim/skew}

\part{Persistência}
\subfile{parts/persist/intro}

\subfile{parts/persist/pilha}
\subfile{parts/persist/deque1}

\subfile{parts/persist/deque2}
\subfile{parts/persist/deque3}

\subfile{parts/persist/geral}
\subfile{parts/persist/rubro-negra}

\subfile{parts/persist/point-location}

%\part{Retroatividade}
%\subfile{parts/retro/intro}
%
%\subfile{parts/retro/fila}

\bibliographystyle{plain}
\bibliography{main}

\end{document}
