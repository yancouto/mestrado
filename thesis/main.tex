\documentclass[11pt,oneside,a4paper, openany]{book}

% Meus .sty
\usepackage{formatting}
\usepackage{code}
\usepackage{theorem-vars}
\usepackage{drawing}

\title{Estruturas de dados persistentes}
\author{Yan Couto}
\date{\today}
\floatstyle{plain}
\newfloat{example}{tbhp}{loe}[chapter]
\floatname{example}{Exemplo}

\begin{document}

\frontmatter

% ---------------------------------------------------------------------------- %
% CAPA
% Nota: O título para as dissertações/teses do IME-USP devem caber em um 
% orifício de 10,7cm de largura x 6,0cm de altura que há na capa fornecida pela SPG.
\thispagestyle{empty}
\begin{center}
    \vspace*{2.3cm}
    \textbf{\Large{Estruturas de dados persistentes}}\\

    \vspace*{1.2cm}
    \Large{Yan Soares Couto}

    \vskip 2cm
    \textsc{
    Dissertação apresentada\\[-0.25cm] 
    ao\\[-0.25cm]
    Instituto de Matemática e Estatística\\[-0.25cm]
    da\\[-0.25cm]
    Universidade de São Paulo\\[-0.25cm]
    para\\[-0.25cm]
    obtenção do título\\[-0.25cm]
    de\\[-0.25cm]
    Mestre em Ciências}

    \vskip 1.5cm
    Programa: Ciência da Computação\\
    Orientadora: Profa. Dra. Cristina Gomes Fernandes

    \vskip 1cm
    \normalsize{Durante o desenvolvimento deste trabalho o autor recebeu auxílio financeiro da Fundação de Amparo à Pesquisa do Estado de São Paulo (FAPESP), processo no 2017/05481-8, e da Coordenação de Aperfeiçoamento de Pessoal de Nível Superior (CAPES).}

    \vskip 0.5cm
    \normalsize{São Paulo, dezembro de 2018}
\end{center}

% ---------------------------------------------------------------------------- %
% Página de rosto (SÓ PARA A VERSÃO DEPOSITADA - ANTES DA DEFESA)
% Resolução CoPGr 5890 (20/12/2010)
%
% IMPORTANTE:
%   Coloque um '%' em todas as linhas
%   desta página antes de compilar a versão
%   final, corrigida, do trabalho
%
%
\newpage
\thispagestyle{empty}
    \begin{center}
        \vspace*{2.3 cm}
        \textbf{\Large{Estruturas de dados persistentes}}\\
        \vspace*{2 cm}
    \end{center}

    \vskip 2cm

    \begin{flushright}
	Este exemplar corresponde à redação \\
	final da dissertação devidamente corrigida, \\
	defendida por Yan Soares Couto \\
	e aprovada pela Comissão Julgadora.
    \end{flushright}

    \vskip 2cm

    \begin{flushright}
	As opiniões, hipóteses e conclusões \\
	ou recomendações expressas neste \\
	material são de responsabilidade do \\
	autor e não necessariamente refletem \\
	a visão da FAPESP e da CAPES.
    \end{flushright}

\pagebreak

% ---------------------------------------------------------------------------- %
% Resumo
\chapter*{Resumo}

\noindent COUTO, Y. S. \textbf{Estruturas de dados persistentes}.
2018. 80 f.
Dissertação (Mestrado) - Instituto de Matemática e Estatística,
Universidade de São Paulo, São Paulo, 2018.
\\

Estruturas de dados (EDs) permitem operações de acesso e de modificação; operações de acesso apenas consultam um ou mais campos de uma ED, enquanto operações de modificação podem alterar os campos da estrutura. Dizemos que, ao realizar uma operação de modificação, criamos uma nova versão da ED.

Uma ED é \deff{parcialmente persistente} se permite apenas operações de acesso a versões anteriores e modificação apenas na versão mais nova, e \deff{totalmente persistente} se também permite operações de modificação em todas as versões.

Esta dissertação apresenta a descrição e implementação de uma versão total ou parcialmente persistente de várias estruturas: pilhas, filas, deques e árvores rubro-negras. Também são discutidas técnicas gerais para tornar persistentes certas classes de estruturas de dados. Por fim, é apresentada uma solução ao problema de localização de ponto, que usa uma árvore de busca binária persistente.
\\

\noindent \textbf{Palavras-chave:} estruturas de dados, persistência, árvores rubro-negras, localização de ponto

% ---------------------------------------------------------------------------- %
% Abstract
\chapter*{Abstract}
\noindent Couto, Y. S. \textbf{Persistent data structures}. 
2018. 80 f.
Dissertação (Mestrado) - Instituto de Matemática e Estatística,
Universidade de São Paulo, São Paulo, 2018.
\\

Data structures (DSs) allow access and update operations; access operations only allow accessing the value of one or more fields of the DS, while update operations allow modifying the fields of the structure. We say that, whenever an update operation is done, a new version of the DS is created.

A DS is~\deff{partially persistent} if it allows access operations to previous versions of the structure and update operations only on the newest version, and~\deff{totally persistent} if it also allows update operations on all versions.

This dissertation presents the description and implementation of totally or partially persistent versions of several data structures: stacks, queues, deques, and red-black trees. General techniques to make certain classes of DSs persistent are also discussed. At last, a solution to the point location problem, using a persistent binary search tree, is presented.
\\

\noindent \textbf{Keywords:} data structures, persistence, red-black trees, point location


\setcounter{tocdepth}{1}

\pagenumbering{gobble}
\begingroup
\let\cleardoublepage\clearpage
\tableofcontents
\endgroup

\mainmatter
\pagenumbering{arabic}

\subfile{intro}

\part{Preliminares} \label{part:prelim}
%\subfile{parts/prelim/intro}

\subfile{parts/prelim/ancestrais}

\subfile{parts/prelim/skew}

\part{Persistência} \label{part:persist}
%\subfile{parts/persist/intro}

\subfile{parts/persist/pilha}
\subfile{parts/persist/deque1}

\subfile{parts/persist/deque2}
\subfile{parts/persist/deque3}

\subfile{parts/persist/geral}
\subfile{parts/persist/rubro-negra}

\subfile{parts/persist/point-location}

%\part{Retroatividade}
%\subfile{parts/retro/intro}
%
%\subfile{parts/retro/fila}

\subfile{parts/persist/conclusao}

\bibliographystyle{plain}
\bibliography{main}

\end{document}
