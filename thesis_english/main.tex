\documentclass[11pt,oneside,a4paper, openany]{book}

% My .sty
\usepackage{formatting}
\usepackage{code}
\usepackage{theorem-vars}
\usepackage{drawing}

\title{Persistent data structures}
\author{Yan Couto}
\date{\today}
\floatstyle{plain}
\newfloat{example}{tbhp}{loe}[chapter]
\floatname{example}{Example}

\begin{document}

\frontmatter

% ---------------------------------------------------------------------------- %
% CAPA
\thispagestyle{empty}
\begin{center}
    \vspace*{2.3cm}
    \textbf{\Large{Persistent data structures}}\\

    \vspace*{1.2cm}
    \Large{Yan Soares Couto}

    \vskip 2cm

    \vskip 1.5cm
    Program: Computer Science\\
    Advisor: Profa. Dra. Cristina Gomes Fernandes

    \vskip 1cm

    \vskip 0.5cm
\end{center}


% ---------------------------------------------------------------------------- %
% Abstract
\chapter*{Abstract}
\noindent Couto, Y. S. \textbf{Persistent data structures}. 
2018. 80 f.
Dissertação (Mestrado) - Instituto de Matemática e Estatística,
Universidade de São Paulo, São Paulo, 2018.
\\

Data structures (DSs) allow access and update operations; access operations only allow accessing the value of one or more fields of the DS, while update operations allow modifying the fields of the structure. We say that, whenever an update operation is done, a new version of the DS is created.

A DS is~\deff{partially persistent} if it allows access operations to previous versions of the structure and update operations only on the newest version, and~\deff{totally persistent} if it also allows update operations on all versions.

This dissertation presents the description and implementation of totally or partially persistent versions of several data structures: stacks, queues, deques, and red-black trees. General techniques to make certain classes of DSs persistent are also discussed. At last, a solution to the point location problem, using a persistent binary search tree, is presented.
\\

\noindent \textbf{Keywords:} data structures, persistence, red-black trees, point location


\setcounter{tocdepth}{1}

\pagenumbering{gobble}
\begingroup
\let\cleardoublepage\clearpage
\tableofcontents
\endgroup

\mainmatter
\pagenumbering{arabic}

%\subfile{intro}

\part{Preliminary} \label{part:prelim}
%\subfile{parts/prelim/intro}

\subfile{parts/prelim/ancestors}

\subfile{parts/prelim/skew}

\part{Persistence} \label{part:persist}

Untranslated
%\subfile{parts/persist/intro}

%\subfile{parts/persist/pilha}
%\subfile{parts/persist/deque1}

%\subfile{parts/persist/deque2}
%\subfile{parts/persist/deque3}

%\subfile{parts/persist/geral}
%\subfile{parts/persist/rubro-negra}

%\subfile{parts/persist/point-location}

%\part{Retroatividade}
%\subfile{parts/retro/intro}
%
%\subfile{parts/retro/fila}

%\subfile{parts/persist/conclusao}

\bibliographystyle{plain}
\bibliography{main}

\end{document}
